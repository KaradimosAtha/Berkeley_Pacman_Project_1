\documentclass{article}
\usepackage{graphicx} % Required for inserting images
\usepackage[left=2cm,right=1cm]{geometry}
\usepackage[greek,english]{babel}
\usepackage{alphabeta}
\usepackage{relsize}
\usepackage{amsmath}
\usepackage{graphicx}
\title{\underline{README PACMAN PROJECT 1}}
\author{\LargeΑθανάσιος Καραδήμος(ΑΜ: 1115202300062)}
\date{Οκτώβριος 2024}


\begin{document}

\maketitle

\textbf{\underline{\LargeΣχεδιαστικές Επιλογές}}\\
\newline
$\bullet$\textbf{\Large \underline{Question 5}}\\
\newline
\Large Στο Corners Problem η κατασταση που χρησιμοποιω ειναι ενα tuple που περιεχει τις συντεταγμενες θεσης του pacman $(x,y)$ και ενα tuple που αποτελειται απο ολα τα ζευγαρια συντεταγμενων(x,y) των corners που πρεπει να επισκευτει ο pacman.\\
\newline
Για να καταλαβουμε εαν μια κατασταση ειναι Goal State θα πρεπει το tuple που περιεχει τα corners να εχει μεγεθος $0$ , αυτο θα σημαινει οτι επισκευτικαμε επιτυχως και τις 4 γωνιες και ο αλγοριθμος θα σταματισει. \\
\newline
Η ενημερωση του tuple γινεται στην συναρτηση getSuccessors , μετατρεπωντας το σε λιστα για να αλλαξουμε τα περιεχομενα του , καθως τα tuples ειναι immutable . Επισης καθως περνουμε τον καθε successor πρεπει να ελεγχουμε εαν ειναι κατασταση στοχου δηλαδη εαν οι συντεταγμενες του αντιστοιχουν σε καποια απο τα corners που δεν εχουμε κανει visit. Εαν ισχυει αυτο τοτε αφαιρουμε απο την λιστα των unvisited corners την συγκεκριμενη γωνια. Σε καθε περιπτωση η getSuccessors θα επιστρεψει μια λιστα απο tuples οπου το καθενα θα περιεχει τον successor , το action και το κοστος μονοπατιου , το successor θα ειναι ενα tuple με πρωτο στοιχειο τις συντεταγμενες του και δευτερο την ενημερωμενη ή οχι λιστα του state(αναλογα με το εαν οι συντεταγμενες του successor ανηκουν στο tuple των corners) την οποια θα την εχουμε μετατρεψει παλι σε tuple μεσο της συναρτησης tuple(). \\
\newpage
$\bullet$\textbf{\Large \underline{Question 6}}\\
\newline
Στην ευρετικη συναρτηση του προληματος $6$ υπολογιζω αρχικα ολα τα manhattan distances απο το position στο οποιο βρισκομαι , προς ολα τα corners που δεν εχω κανει visit και τα αποθηκευω σε μια λιστα. Στοχος μου ειναι να γυρισω μια τιμη η οποια να μην ειναι παρα μεγαλη ετσι ωστε η συναρτηση να μην ειναι admissible αλλα να γυρισω μια η οποια να βρισκεται στην μεση απο ολες τις αποστασεις που υπολογισα. Ο τυπος υπολογισμου δεν προηλθε απο καπου συγκεκριμενα , απλος εκανα αρκετες δοκιμες μεχρι να τον βρω. Πρακτικα υπολογιζω τον μεσο ορο της λιστας που εχω αποθηκευσει τα manhattan distances ,τον προσθετω 
στην μεγιστη αποσταση , και τελος αφαιρω την ελαχιστη αποσταση.\\
\newline
Δηλαδη: $h = (max + average) - min$ \\
\newline
Φυσικα εαν το μεγεθος της λιστας ειναι $0$ τοτε επιστρεφουμε $0$ γιατι εχουμε επισκευτει ολσ τα corners(δεν θα μπουμε ποτε στο for loop , δηλαδη δεν θα υπαρχει στοιχειο που να ανηκει στο tuple των unvisited corners, οποτε η λιστα θα παραμηνει κενη).\\
\newline
$\bullet$\textbf{\Large \underline{Question 7}}\\
\newline
Στην ευρετικη συναρτηση του προβληματος $7$ υπολογιζω οπως και στο προβλημα $6$ ολα τα manhattan distances απο το position που βρισκομαι προς ολα τα foods που δεν εχει φαει ο pacman , με την διαφορα οτι υπολογιζω και την μεγιστη αποσταση αλλα και τις συντεταγμενες του φαγητου απο το οποιο εχω αυτο max distance.Στην συνεχεια για αυτη την αποσταση υπολογιζω ποσο ειναι η πραγματικη μεσω της συναρτησης maze distance.\\
\newline
Τελος επιστρεφω: $h = |$max manhattan$ - $maze distance$| + $max manhattan$ $\\
\newline
Εαν επιστεψουμε μονο την αφαιρεση της μεγιστης αποστασης manhattan απο την πραγματικη αποσταση , τοτε το αποτελεσμα μπορει να ειναι 0 ή παρα πολυ μικρο , για αυτο προσθετω μετα παλι το max manhattan.\\
\newline
Εαν το μεγεθος της λιστας που αποθηκευουμαι τα distances ειναι 0 τοτε επιστρεφουμε 0 διοτι σημαινει οτι εχουμε φαει ολα τα φαγητα.\\
\newline
Τελος οπως και στο $q_6$ η συναρτηση αυτη προεκιψε μετα απο πολλες δοκιμες.\\
\newline
$\bullet$\textbf{\Large \underline{Question 8}}\\
\newline
Στην ερωτηση $8$ εφοσον θελουμε να βρουμε ενα μονοπατι προς το κοντινοτερο φαγητο, καλουμε την συναρτηση BFS που υλοποιησαμε στην ερωτηση 2 , περνωντας της σαν ορισμα το προβλημα που μας δινεται. Ο BFS θα βρει σιγουρα πρωτα την πιο κοντινη κουκιδα γιατι εξεταζει ανα επιπεδα , απλος μπορει η διαδρομη που θα κανει για να φτασει σε αυτη να μην ειναι η πιο βελτιστη για να το βοηθησει μετα στο να φαει και ολες τις υπολοιπες.\\
\newline
Για να ελεγξουμε εαν μια κατασταση ειναι κατασταση στοχου ειναι πολυ ευκολο εφοσον μας δινεται ενας πινακας που στην θεση $(x,y)$ περιεχει την τιμη True εαν υπαρχει φαγητο, αλλιως την τιμη False, απλος πρεπει να ελεγξουμε εαν στην θεση (positionx ,positiony) του πινακα εχουμε τιμη True.(οπου positionx και positiony ειναι οι συντεταγμενες της καταστασης που βρισκομαστε στο grid.)\\
\newline
$\bullet$\textbf{\Large \underline{Question 1,2,3,4}}\\
\newline
Δεν εχω να προσθεσω κατι σχετικα με τις 4 πρωτες ερωτησεις , ακολουθησα τις διαφανειες του φροντιστηριου.\\
\newline
\end{document}
