\documentclass{article}
\usepackage{graphicx} % Required for inserting images
\usepackage[left=2cm,right=1cm]{geometry}
\usepackage[greek,english]{babel}
\usepackage{alphabeta}
\usepackage{relsize}
\usepackage{amsmath}
\usepackage{graphicx}
\title{ \huge\underline{Εργασία 1 AI} }
\author{\LargeΑθανάσιος Καραδήμος(ΑΜ: 1115202300062)}
\date{Οκτώβριος 2024}

\begin{document}


\maketitle

\textbf{\Large\textit{\underline{Πρόβλημα 2}}:}
\\
\newline
\Large Θεωρω οτι ενας κομβος επεκτινεται οταν γινεται ο ελεγχος (αμεσως μετα αφου τον βγαλουμε απο το frontier)εαν ειναι κατασταση στοχου. Εαν ειναι τοτε δεν τον μετραω σαν επεκτεταμενο κομβο , αλλιως εαν δεν ειναι τοτε μετριεται κανονικα. Δηλαδη εαν εχω εναν κομβο A ο οποιος εχει σαν παιδια το B,C,D και ο D ειναι κατασταση στοχου , τοτε επεκτηνω 3 κομβους τον A,B,C . Ο D δεν επεκτινεται γιατι ειναι κατασταση στοχου και ο αλγοριθμος θα σταματησει χωρις να βαλει τα παιδια του στο frontier.(Η αριθμηση για τα βαθη ξεκιναει απο το 0)\\ 
\newline
\newline
\textbf{\Large 1)\underline{BFS:}}\\ 
\newline 
\newline
    \textbf{\Large\hspace{0.5cm}\underline{Περιπτωση 1:}}\\ 
    \newline
    \LargeΕαν καθως βρισκουμε τους successors ενος κομβου δεν ελεγχουμε εαν καποιος απο αυτους ειναι κατασταση στοχου(δηλαδη ο ελεγχος για το εαν μια κατασταση ειναι κατασταση στοχου γινεται μονο οταν βγαζουμε κομβο απο το fringe) τοτε:\\
    \newline
    $\bullet$\LargeΜικρότερος αριθμός κομβων που επεκτίνονται: $40$\\
    $\bullet$\LargeΜέγιστος αριθμός κοόβων που επεκτίνονται: $120$\\
    \newline
    \Large\underline{Επεξήση:} \\ 
    \newline
    \LargeΗ "καλυτερη" θεση στην οποια μπορει να βρισκεται η κατασταση στοχου σε βαθος 4 ειναι στο πρωτο παιδι του του πρωτου κομβου απο το βαθος 3(δηλαδη ο αριστεροτερος κομβος απο το βαθος 4) . Επειδη ο BFS εξεταζει ολους τους κομβους του δεντρου ανα επιπεδα , για να φτασουμε στον ζητουμενο κομβο θα πρεπει να περασουμε απο $40$ κομβους , διοτι σε βαθος $0->1$ επεκτινουμε 1 κομβο , σε βαθος $1 -> 3$  , σε βαθος $2 ->9$ , σε βαθος $3->27$ και ο ακριβως επομενος κομβος στο βαθος $4$ θα ειναι ο ζητουμενος. Αρα συνολικα πρεπει να επεκτηνουμε $1 + 3 + 9 + 27 = 40 \longleftrightarrow b^0 + b^1 + b^2 + b^3$(Οπου $b = 3$ το branching factor) κομβους. Καποιος θα μπορουσε να πει 41 κομβους αλλα θεωρω οτι οταν φτασουμε στην κατασταση στοχου τοτε αυτος ο κομβος δεν επεκτεινεται.\\
    \newline
    \LargeΗ "χειρότερη θεση στην οποια μπορει να βρισκεται η κατασταση στοχου σε βαθος 4 ειναι στο τελευταιο παιδι του τελευταιου κομβου απο το βαθος 3(δηλαδη ο τερμα δεξια κομβος του βαθους 4) . Ετσι ο BFS θα πρεπει να βαλει τα παιδια ολων τον προηγουμενων κομβων στο frontier μεχρι να φτασει σε εκεινον , με αποτελεσμα να πρεπει να περασει απο $1 + 3 + 9 + 27 + 80 = 120$ επεκταμενους κομβους.\\
    \newline
    \textbf{\Large\hspace{0.5cm}\underline{Περιπτωση 2:}}\\ 
    \newline
    \LargeΕαν καθως βρισκουμε τους successors ενος κομβου ελεγχουμε ταυτοχρονα εαν καποιος απο αυτους ειναι κατασταση στοχου. Κανοντας αυτο το optimization εχουμε οτι:\\
    \newline
    $\bullet$\LargeΜικρότερος αριθμός κομβων που επεκτίνονται: $14$\\
    $\bullet$\LargeΜέγιστος αριθμός κοόβων που επεκτίνονται: $40$\\
    \newline
    \Large\underline{Επεξήγηση:}\\
    \newline
    \LargeΣτην καλυτερη περιπτωση θα εχουμε $14$ πληρους εκτεταμενους κομβους
    ($3^{0} + 3^{1} + 3^{2} + 1)$ γιατι μολις ο BFS φτασει στον κομβο που εχει σαν παιδι την κατασταση στοχου τοτε με το που θα γινει ο ελεγχος για το εαν το παιδι ειναι κατασταση στοχου , ο αλγοριθμος θα σταματησει. Ο πατερας αυτου το κομβου θα εχει επεκταθει καθως ελεγχουμε στην αρχη εαν ειναι Goal state. \\
    \newline
    \LargeΣτην χειροτερη περιπτωση θα εχουμε επεκτηνει $40$ κομβους($3^{0} + 3^{1} + 3^{2} + 3^{3}$) . Μολις ο BFS φτασει στον τελευταιο κομβο απο το επιπεδο $3$ θα γινει ο ελεγχος οτι το τελευταιο παιδι του ειναι η κατασταση στοχου και ο αλγοριθμος θα σταματησει. \\
    \newline
    \newline
    
    \textbf{\Large 2)\underline{DFS}:}\\ 
    \newline 
    \Large\hspace{0.5cm}
    \LargeΣτην περιπτωση του DFS η καλυτερη περιπτωση ειναι η κατασταση στοχου να βρισκεται στον αριστεροτερο κομβο σε βαθος 4. Τοτε θα εχουμε επεκτηνει $4$ κομβους , γιατι θα εχουμε ελεγξει εαν ειναι goal state ο πρωτος κομβος στο βαθος 0 , μετα ο πρωτος κομβος(μετρωντας απο τα αριστερα) στο βαθος 1 , το ιδιο για το βαθος 2 και 3 και στο 4 θα ειναι ο ζητουμενος κομβος οπου δεν τον επεκτηνουμε γιατι ειναι κατασταση στοχου. \\
    \newline
    \LargeΕαν η κατασταση στοχου βρισκεται οπουδηποτε αλλου στο τεταρτο επιπεδο και ο χωρος καταστασεων ειναι απειρος τοτε ο DFS θα πηγενει ολο και πιο βαθια εξεταζωντας καθε φορα το αριστερο παιδι καθε επιπεδου με αποτελεσμα ο αλγοριθμος να μην φτασει ποτε στην κατασταση στοχου.\\ 
    \newline
    \LargeΕαν ο χωρος καταστασεων δεν ηταν απειρος , πχ το δεντρο ειχε μεγιστο βαθος $6$ τοτε θα ειχαμε επεκτινει συνολικα $3^{0} + 3^{1} + 3^{2} + 3^{3} + 3^{4} + 3^{5} + 3^{6} - 13 = 1080$ κομβους. Το 13 το αφαιρουμε γιατι δεν επεκτινουμε την κατασταση στοχου , τα 3 παιδια της και τα 9 συνολικα παιδια των παιδιων . Οποτε ο γενικος τυπος expanded nodes οταν ο χωρος καταστεσεων δεν ειναι απειρος και η κατασταση στοχου ειναι ο δεξιοτερος κομβος στο βαθος 4 ειναι: $3^{0} + 3^{1} + 3^{2} + .... + 3^{m} - (3^{0} + 3^{1} + ... + 3^{m-4})$ , οπου m ειναι το μεγιστο βαθος.
    (Στον τυπο προσθετουμε ολους του κομβους απο το καθε επιπεδο και στο τελος αφαιρουμε ολους τους κομβους που προκειπτουν ξεκινωντας DFS απο το goal state στο βαθος 4(για αυτο βαζουμε m-4) )
    \newline
    \newline
    \newline
    \newline
     \textbf{\Large 3)\underline{IDS}:}\\
    \newline 
    \newline
    \Large\hspace{0.5cm} 
    \LargeΣτον αλγοριθμο IDS η καλυτερη περιπτωση ειναι παλι η κατασταση στοχου να βρισκεται στον αριστεροτερο κομβο του τεταρτου επιπεδου. Τοτε με βαση τον IDS θα εχουμε επεκτεινει συνολικα $1 + 4 + 13 + 40 + 4 = 62$ κομβους.\\
    \newline
     \Large\underline{Επεξήγηση:}\\
      \newline
      \LargeΟ αλγοριθμος θα ξεκινησει αρχικα με βαθος $0$ , οπου εκει επεκτηνουμε το start node. Στην συνεχεια θα παμε σε βαθος $1$ οπου ξεκινώντας απο την αρχη μπορουμε να επεκτηνουμε $4$ κομβους(τον αρχικο και τα $3$ παιδια του) . Μετα θα παμε σε βαθος $2$ οπου ξεκινωντας παλι απο την αρχη και κανωντας DFS θα μπορουμε να επεκτηνουμε $13$ κομβους(Τον πρωτο , τα τρια του παιδια και τα παιδια των παιδιων που ειναι $3^{2}$) . Οταν φτασουμε στο βαθος $3$ θα μπορουμε να επεκτηνουμε $40 -> 3^{0} + 3^{1} + 3^{2} + 3^{3}$ κομβους(παλι μεσω DFS) . Τελος οταν θα παμε σε βαθος 4 οπου στον αριστεροτερο κομβο βρισκεται το Goal State , θα κανουμε DFS ξεκινωντας απο την αρχικη κατασταση , μετα θα επεκτηνουμε συνεχεια αριστερα παιδια μεχρι να βρουμε την κατασταση στοχου , αρα θα εχουμε επεκτηνει 4 κομβους . Οποτε συνολικα $62 = 3^{0} + 3^{1} + 3^{2} + 3^{3} + 4$. \\
      \newline
      \newline
      \LargeΟσον αφορα την χειροτερη περιπτωση , στην οποια το goal state ειναι ο δεξιοτερος κομβος του 4ου επιπεδου , θα εχουμε επεκτηνει συνολικα $1 + 4 + 13 + 40 + 40 + 3^{4} -1 = 178$ κομβους εαν ακολουθησουμε τα ιδια βηματα με πανω , μονο που οταν θα μπορουμε να παμε μεχρι βαθος 4 δεν θα επεκτηνουμε μονο $4$ κομβους αλλα $3^{0} + 3^{1} + 3^{2} + 3^{3} + 3^{4} - 1 = 120$ διοτι κανοντας DFS ξεκινωντας απο την ριζα θα επεκτινουμε οπως πριν ολους τους κομβους μεχρι το επιπεδο $3$ και στην συνεχεια ολους τους κομβους στο επιπεδο $4$ που ειναι $3^{4}$ αφαιρωντας ομως τον κομβο που ειναι goal state(γιατι οπως εχω πει και στην αρχη του προβληματος $2$ , δεν τον μετραω σαν expanded node) .Οποτε εχουμε $120$ κομβους οταν μπορουμε να παμε μεχρι το βαθος $4$ $ + $ ολους τους προηγουμενους κομβους που επεκτηνε ο IDS οταν μπορουσε να παει αρχικα μεχρι βαθος $0$ μετα $1$ , $2$ , $3$ που ειναι συνολικα $58$ αρα εχουμε κανει expand $120 + 58 = 178$ nodes .
      \newline
      \newline
      \textbf{\Large\textit{\underline{Πρόβλημα 3}}:}\\
      \newline
      \newline
      Π: Πάνω\\
      Κ: Κάτω \\
      Α: Αριστερά\\
      Δ: Δεξιά\\
      \textbf{Θεωρω οτι για να παω στο G εχω κοστος 1 και στο τελος το G δεν επεκτινεται}\\
      \newline
      $\bullet$\hspace{0.1cm}Εισαγεται ο S($g= 0 , h = 8$):\newline
      \vspace{-150cm}
      \vspace*{150cm}Fringe: [ $(S,8$) ]\newline
      \vspace{-150cm}
      \vspace*{150cm}
      Explored: [ ]
      \newline
      \newline
      $\bullet$\hspace{0.1cm}Αφαιρειται ο S και εισαγονται οι γειτονες του  $  ((1,0) , 8.5(g:1,h:7.5) ,$Κ$) ,( (0,1),\\8.5(g:1,h:7.5) ,$ Δ$)$:\newline
     \vspace{-150cm}
     \vspace*{150cm}Fringe: [ $( (1,0) , 8.5, $Κ$)\hspace{0.3cm} \vert\hspace{0.3cm} ( (0,1) , 8.5 , $Δ$ )$ ]\newline
     \vspace{-150cm}
     \vspace*{150cm}
     Explored: [$S$]\newline
     \newline
    $\bullet$\hspace{0.1cm}Αφαιρειται ο $(1,0)$ (που ειναι η Κατω ενεργεια του ρομποτ απο τον κομβο S) και εισαγονται οι γειτονες του $((2,0) , 9(g:2,h:7) ,$Κ$)$:
    \newline
     \vspace{-150cm}
     \vspace*{150cm}Fringe: [ $( (0,1) , 8.5, $Δ$)\hspace{0.3cm} \vert\hspace{0.3cm} ((2,0) , 9, $Κ$)$ ]\newline
     \vspace{-150cm}
     \vspace*{150cm}
     Explored: [$S , (1,0)$]
     \newpage
     $\bullet$\hspace{0.1cm}Αφαιρειται ο $(0,1)$ (που ειναι η Δεξια ενεργεια του ρομποτ απο τον κομβο S) και εισαγονται οι γειτονες του $((0,2) , 9(g:2,h:7) ,$Δ$)$:
    \newline
     \vspace{-150cm}
     \vspace*{150cm}Fringe: [ $((2,0) , 9 , $Κ$) \hspace{0.3cm} \vert\hspace{0.3cm} (0,2),9 ,$Δ$)$ ]
     \newline
     \vspace{-150cm}
     \vspace*{150cm}
     Explored: [$S , (1,0) , (0,1)$]\newline
     \newline
    $\bullet$\hspace{0.1cm}Αφαιρειται ο $(2,0)$ (που ειναι η Κατω ενεργεια του ρομποτ απο τον κομβο $(1,0)$) και εισαγονται οι γειτονες του $((3,0) , 9.5(g:3,h:6.5) ,$Κ$) , ((2,1) ,10.5(g:4 , h:6.5) , $Δ$)$:
    \newline
     \vspace{-150cm}
     \vspace*{150cm}Fringe: [ $( (0,2),9 , $Δ$ ) \hspace{0.3cm} \vert\hspace{0.3cm} ((3,0) , 9.5,$Κ$) \hspace{0.3cm} \vert\hspace{0.3cm} ( (2,1) ,10.5, $Δ$)$ ]
     \newline
     \vspace{-150cm}
     \vspace*{150cm}
     Explored: [$S , (1,0) , (0,1) , (2,0) $]\newline
     \newline
     $\bullet$\hspace{0.1cm}Αφαιρειται ο $(0,2)$ (που ειναι η Δεξια ενεργεια του ρομποτ απο τον κομβο $(0,1)$) και εισαγονται οι γειτονες του $((0,3) , 9.5(g:3,h:6.5) ,$Δ$)$:
    \newline
     \vspace{-150cm}
     \vspace*{150cm}Fringe: [ $((3,0) , 9.5,$Κ$) \hspace{0.3cm} \vert\hspace{0.3cm} (0,3) , 9.5,$Δ$) \hspace{0.3cm} \vert\hspace{0.3cm} ( (2,1) ,10.5, $Δ$)$ ]
     \newline
     \vspace{-150cm}
     \vspace*{150cm}
     Explored: [$S , (1,0) , (0,1) , (2,0) , (0,2) $]\newline
     \newline
     $\bullet$\hspace{0.1cm}Αφαιρειται ο $(3,0)$ (που ειναι η Κατω ενεργεια του ρομποτ απο τον κομβο $(2,0)$) και εισαγονται οι γειτονες του $((4,0) , 10 ,$Κ$) , ( (3,1) , 10 , $Δ$ )$:
    \newline
     \vspace{-150cm}
     \vspace*{150cm}Fringe: [ $ (0,3) , 9.5,$Δ$) \hspace{0.3cm} \vert\hspace{0.3cm} (4,0) , 10 ,$Κ$) \hspace{0.3cm} \vert\hspace{0.3cm} ( (3,1) , 10 , $Δ$ ) \hspace{0.3cm} \vert\hspace{0.3cm} ( (2,1) ,10.5, $Δ$)$ ]
     \newline
     \vspace{-150cm}
     \vspace*{150cm}
     Explored: [$S , (1,0) , (0,1) , (2,0) , (0,2) , (3,0) $]\newline
     \newline
     $\bullet$\hspace{0.1cm}Αφαιρειται ο $(0,3)$ (που ειναι η Δεξια ενεργεια του ρομποτ απο τον κομβο $(0,2)$) και εισαγονται οι γειτονες του $((1,3) , 10 ,$Κ$) $:
    \newline
     \vspace{-150cm}
     \vspace*{150cm}Fringe: [ $ (4,0) , 10 ,$Κ$) \hspace{0.3cm} \vert\hspace{0.3cm} ( (3,1) , 10 , $Δ$ ) \hspace{0.3cm} \vert\hspace{0.3cm} ((1,3) , 10 ,$Κ$) \hspace{0.3cm} \vert\hspace{0.3cm} ( (2,1) ,10.5, $Δ$)$ ]
     \newline
     \vspace{-150cm}
     \vspace*{150cm}
     Explored: [$S , (1,0) , (0,1) , (2,0) , (0,2) , (3,0) , (0,3) $]\newline
     \newline
    $\bullet$\hspace{0.1cm}Αφαιρειται ο $(4,0)$ (που ειναι η Κατω ενεργεια του ρομποτ απο τον κομβο $(3,0)$) και εισαγονται οι γειτονες του $((4,1) , 10.5 ,$Δ$) $:
    \newline
     \vspace{-150cm}
     \vspace*{150cm}Fringe: [ $ ( (3,1) , 10 , $Δ$ ) \hspace{0.3cm} \vert\hspace{0.3cm} ((1,3) , 10 ,$Κ$) \hspace{0.3cm} \vert\hspace{0.3cm} ( (2,1) ,10.5, $Δ$) \hspace{0.3cm} \vert\hspace{0.3cm}((4,1) , 10.5 ,$Δ$) $ ]
     \newline
     \vspace{-150cm}
     \vspace*{150cm}
     Explored: [$S , (1,0) , (0,1) , (2,0) , (0,2) , (3,0) , (0,3) ,(4,0) $]\newline
    \newline
    \newpage
    $\bullet$\hspace{0.1cm}Αφαιρειται ο $(3,1)$ (που ειναι η Δεξια ενεργεια του ρομποτ απο τον κομβο $(3,0)$) και εισαγονται οι γειτονες του , μπορουμε να παμε στον $(2,1)$ αλλα με κοστος $f = 12.5(g:6 , h:6.5) > 10.5$ αρα δεν ανταλασουμε το $(2,1)$ μεσα στο fringe με το  $(2,1)$ που προηλθε απο τον κομβο $(3,1)$ . Ομοιος και για το $(4,1)$ το οποιο μεσω του $(3,1)$ εχει $f = 10.5 = 10.5$ του $(4,1)$ μεσα στο fringe , αρα δεν το ανταλλασουμε . Αρα ο μονος γειτονας που εισαγουμε ειναι ο $((3,2) , 10.5 , $Δ$ ) $
    \newline
     \vspace{-150cm}
     \vspace*{150cm}Fringe: [ $ ((1,3) , 10 ,$Κ$) \hspace{0.3cm} \vert\hspace{0.3cm} ((2,1) ,10.5, $Δ$) \hspace{0.3cm} \vert\hspace{0.3cm}((4,1) , 10.5 ,$Δ$)\hspace{0.3cm} \vert\hspace{0.3cm} ((3,2) , 10.5 , $Δ$ )  $ ]
     \newline
     \vspace{-150cm}
     \vspace*{150cm}
     Explored: [$S , (1,0) , (0,1) , (2,0) , (0,2) , (3,0) , (0,3) ,(4,0) , (3,1) $]\newline
    \newline
     $\bullet$\hspace{0.1cm}Αφαιρειται ο $(1,3)$ (που ειναι η Κατω ενεργεια του ρομποτ απο τον κομβο $(0,3)$) και εισαγονται οι γειτονες του  $((2,3) , 10.5 , $Κ$ ) , ((1,4) , 10 , $Δ$ )$
    \newline
     \vspace{-150cm}
     \vspace*{150cm}Fringe: [ $((1,4) , 10 , $Δ$ ) \hspace{0.3cm} \vert\hspace{0.3cm} ((2,1) ,10.5, $Δ$) \hspace{0.3cm} \vert\hspace{0.3cm}((4,1) , 10.5 ,$Δ$)\hspace{0.3cm} \vert\hspace{0.3cm} ((3,2) , 10.5 , $Δ$ )\hspace{0.3cm}\vert\hspace{0.3cm} \\((2,3),10.5,$Κ$ )$]
     \newline
     \vspace{-150cm}
     \vspace*{150cm}
     Explored: [$S , (1,0) , (0,1) , (2,0) , (0,2) , (3,0) , (0,3) ,(4,0) , (3,1) , (1,3) $]\newline
    \newline
     $\bullet$\hspace{0.1cm}Αφαιρειται ο $(1,4)$ (που ειναι η Δεξια ενεργεια του ρομποτ απο τον κομβο $(1,3)$) και εισαγονται οι γειτονες του  $ ((1,5) , 10 , $Δ$ )$
    \newline
     \vspace{-150cm}
     \vspace*{150cm}Fringe: [ $((1,5) , 10 , $Δ$ ) \hspace{0.3cm} \vert\hspace{0.3cm} ((2,1) ,10.5, $Δ$) \hspace{0.3cm} \vert\hspace{0.3cm}((4,1) , 10.5 ,$Δ$)\hspace{0.3cm} \vert\hspace{0.3cm} ((3,2) , 10.5 , $Δ$ )\hspace{0.3cm}\vert\hspace{0.3cm} \\((2,3),10.5,$Κ$ )$]
     \newline
     \vspace{-150cm}
     \vspace*{150cm}
     Explored: [$S , (1,0) , (0,1) , (2,0) , (0,2) , (3,0) , (0,3) ,(4,0) , (3,1) , (1,3) ,(1,4)$]
    \newline
    \newline
     $\bullet$\hspace{0.1cm}Αφαιρειται ο $(1,5)$ (που ειναι η Δεξια ενεργεια του ρομποτ απο τον κομβο $(1,4)$) και εισαγονται οι γειτονες του  $ ((2,5) , 10.5 , $Κ$ ) , ((1,6) , 10.5, $Δ$ )$
    \newline
     \vspace{-150cm}
     \vspace*{150cm}Fringe: [ $ ((2,1) ,10.5, $Δ$) \hspace{0.3cm} \vert\hspace{0.3cm}((4,1) , 10.5 ,$Δ$)\hspace{0.3cm} \vert\hspace{0.3cm} ((3,2) , 10.5 , $Δ$ )\hspace{0.3cm}\vert\hspace{0.3cm} \\((2,3),10.5,$Κ$ )\hspace{0.3cm}\vert\hspace{0.3cm} ((2,5) , 10.5 , $Κ$ ) \hspace{0.3cm}\vert\hspace{0.3cm} ((1,6) , 10.5, $Δ$ )$]
     \newline
     \vspace{-150cm}
     \vspace*{150cm}
     Explored: [$S , (1,0) , (0,1) , (2,0) , (0,2) , (3,0) , (0,3) ,(4,0) , (3,1) , (1,3) ,(1,4) , (1,5)$]
    \newline
    \newline
    \newpage
     $\bullet$\hspace{0.1cm}Αφαιρειται ο $(2,1)$ (που ειναι η Δεξια ενεργεια του ρομποτ απο τον κομβο $(2,0)$) και εισαγονται οι γειτονες του , δεν ξαναβαζουμε το $(3,1)$ στο fringe καθως οταν το επισκευτικαμε το ειχαμε κανει με κοστος $f = 10$ ενω τωρα θα το επισκευτουμαι με κοστος $f = 11$ . Ο μονος γειτονας που μπορουμε να παμε απο το $(2,1)$ ειναι ο $((2,2) , 12 , $Δ$) $
    \newline
     \vspace{-150cm}
     \vspace*{150cm}Fringe: [ $((4,1) , 10.5 ,$Δ$)\hspace{0.3cm} \vert\hspace{0.3cm} ((3,2) , 10.5 , $Δ$ )\hspace{0.3cm}\vert\hspace{0.3cm} ((2,3),10.5,$Κ$ )\hspace{0.3cm}\vert\hspace{0.3cm} ((2,5) , 10.5 , $Κ$ ) \hspace{0.3cm}\vert\hspace{0.3cm}\\ ((1,6) , 10.5, $Δ$ ) \hspace{0.3cm} \vert\hspace{0.3cm} (2,2) , 12 , $Δ$) $]
     \newline
     \vspace{-150cm}
     \vspace*{150cm}
     Explored: [$S , (1,0) , (0,1) , (2,0) , (0,2) , (3,0) , (0,3) ,(4,0) , (3,1) , (1,3) ,(1,4) , (1,5) , (2,1)$]
    \newline
    \newline
     $\bullet$\hspace{0.1cm}Αφαιρειται ο $(4,1)$ (που ειναι η Δεξια ενεργεια του ρομποτ απο τον κομβο $(4,0)$) και δεν εισαγουμε κανενα γειτονα καθως δεν μπορουμε να παμε πουθενα(Δεν μπορουμε να παμε στο $(4,0)$ και στο $(3,1)$ γιατι θα εχουμε μεγαλυτερο f cost απο αυτο που τους επισκευτικαμε)
    \newline
     \vspace{-150cm}
     \vspace*{150cm}Fringe: [ $((3,2) , 10.5 , $Δ$ )\hspace{0.3cm}\vert\hspace{0.3cm} ((2,3),10.5,$Κ$ )\hspace{0.3cm}\vert\hspace{0.3cm} ((2,5) , 10.5 , $Κ$ ) \hspace{0.3cm}\vert\hspace{0.3cm}((1,6) , 10.5, $Δ$ ) \hspace{0.3cm} \vert\hspace{0.3cm}\\(2,2) , 12 , $Δ$) $]
     \newline
     \vspace{-150cm}
     \vspace*{150cm}
     Explored: [$S , (1,0) , (0,1) , (2,0) , (0,2) , (3,0) , (0,3) ,(4,0) , (3,1) , (1,3) ,(1,4) , (1,5) , (2,1) ,\\ (4,1)$]
    \newline
    \newline
    $\bullet$\hspace{0.1cm}Αφαιρειται ο $(3,2)$ (που ειναι η Δεξια ενεργεια του ρομποτ απο τον κομβο $(3,1)$) και εισαγονται οι γειτονες του  , το $(2,2)$ δεν το ανταλασουμε με αυτο μεσα στο fringe διοτι εαν το επισκευτουμαι μεσο του $(3,2)$ θα εχουμε $f = 7 + 6 = 13 > 12$ , ομοιως και για το $(3,1)$ το οποιο το εχουμε κανει visit. Αρα ο μονος γειτονας που μενει ειναι ο $((3,3) , 11 , $Δ$ )$
    \newline
     \vspace{-150cm}
     \vspace*{150cm}Fringe: [ $((2,3),10.5,$Κ$ )\hspace{0.3cm}\vert\hspace{0.3cm} ((2,5) , 10.5 , $Κ$ ) \hspace{0.3cm}\vert\hspace{0.3cm}((1,6) , 10.5, $Δ$ ) \hspace{0.3cm} \vert\hspace{0.3cm}((3,3) , 11 , $Δ$ )\hspace{0.3cm}\vert\hspace{0.3cm}\\((2,2) , 12 , $Δ$) $]
     \newline
     \vspace{-150cm}
     \vspace*{150cm}
     Explored: [$S , (1,0) , (0,1) , (2,0) , (0,2) , (3,0) , (0,3) ,(4,0) , (3,1) , (1,3) ,(1,4) , (1,5) , (2,1) ,\\ (4,1) , (3,2)$]
    \newline
    \newline
    $\bullet$\hspace{0.1cm}Αφαιρειται ο $(2,3)$ (που ειναι η Κατω ενεργεια του ρομποτ απο τον κομβο $(1,3)$) και εισαγονται οι γειτονες του  , το $(3,3)$ δεν το ανταλασουμε με αυτο μεσα στο fringe διοτι εαν το επισκευτουμαι μεσο του $(2,3)$ θα εχουμε $f = 6 + 5  = 11 = 11$(αυτο μεσα στο fringe) , ομοιως και για το $(2,2)$ το οποιο εαν το κανουμε visit μεσω του $(2,3)$ θα εχουμε $f = 7 + 6 = 13 > 12$ . Αρα δεν εχει μεινει γειτονας που να μπορουμε να εισαγουμε. 
    \newline
     \vspace{-150cm}
     \vspace*{150cm}Fringe: [ $((2,5) , 10.5 , $Κ$ ) \hspace{0.3cm}\vert\hspace{0.3cm}((1,6) , 10.5, $Δ$ ) \hspace{0.3cm} \vert\hspace{0.3cm}((3,3) , 11 , $Δ$ )\hspace{0.3cm}\vert\hspace{0.3cm}((2,2) , 12 , $Δ$) $]
     \newline
     \vspace{-150cm}
     \vspace*{150cm}
     Explored: [$S , (1,0) , (0,1) , (2,0) , (0,2) , (3,0) , (0,3) ,(4,0) , (3,1) , (1,3) ,(1,4) , (1,5) , (2,1) ,\\ (4,1) , (3,2) , (2,3)$]
    \newline
    \newline
    $\bullet$\hspace{0.1cm}Αφαιρειται ο $(2,5)$ (που ειναι η Κατω ενεργεια του ρομποτ απο τον κομβο $(1,5$) και εισαγονται οι γειτονες του $((3,5) ,11,$Κ$ ) , ((2,6) ,11,$Δ$ )  $
    \newline
     \vspace{-150cm}
     \vspace*{150cm}Fringe: [ $((1,6) , 10.5, $Δ$ ) \hspace{0.3cm} \vert\hspace{0.3cm}((3,3) , 11 , $Δ$ )\hspace{0.3cm}\vert\hspace{0.3cm} ((3,5) ,11,$Κ$ )\hspace{0.3cm}\vert\hspace{0.3cm}  ((2,6) ,11,$Δ$ ) \hspace{0.3cm}\vert\hspace{0.3cm}\\ ((2,2) , 12 , $Δ$) $]
     \newline
     \vspace{-150cm}
     \vspace*{150cm}
     Explored: [$S , (1,0) , (0,1) , (2,0) , (0,2) , (3,0) , (0,3) ,(4,0) , (3,1) , (1,3) ,(1,4) , (1,5) , (2,1) ,\\ (4,1) , (3,2) , (2,3) ,(2,5)$]
    \newline
    \newline
    $\bullet$\hspace{0.1cm}Αφαιρειται ο $(1,6)$ (που ειναι η Δεξια ενεργεια του ρομποτ απο τον κομβο $(1,5$) και εισαγονται οι γειτονες του $((0,6) ,12,$Π$ ) , ((1,7) ,11,$Δ$ )  $ το $(2,6)$ δεν το ανταλασουμε με αυτο του frontier γιατι το f cost του μεσω του $(1,6)$ ειναι ισο με $7 + 4 =11$
    \newline
     \vspace{-150cm}
     \vspace*{150cm}Fringe: [ $((3,3) , 11 , $Δ$ )\hspace{0.3cm}\vert\hspace{0.3cm} ((3,5) ,11,$Κ$ )\hspace{0.3cm}\vert\hspace{0.3cm}  ((2,6) ,11,$Δ$ ) \hspace{0.3cm}\vert\hspace{0.3cm}((1,7) ,11,$Δ$ ) \hspace{0.3cm}\vert\hspace{0.3cm} ((2,2) , 12 , $Δ$) \hspace{0.3cm}\vert\hspace{0.3cm}\\ ((0,6) ,12,$Π$ ) $]
     \newline
     \vspace{-150cm}
     \vspace*{150cm}
     Explored: [$S , (1,0) , (0,1) , (2,0) , (0,2) , (3,0) , (0,3) ,(4,0) , (3,1) , (1,3) ,(1,4) , (1,5) , (2,1) ,\\ (4,1) , (3,2) , (2,3) ,(2,5) , (1,6)$]
    \newline
    \newline
    $\bullet$\hspace{0.1cm}Αφαιρειται ο $(3,3)$ (που ειναι η Δεξια ενεργεια του ρομποτ απο τον κομβο $(3,2$) και εισαγονται οι γειτονες του $((4,3) ,11.5,$Κ$ )$
    \newline
     \vspace{-150cm}
     \vspace*{150cm}Fringe: [ $((3,5) ,11,$Κ$ )\hspace{0.3cm}\vert\hspace{0.3cm}  ((2,6) ,11,$Δ$ ) \hspace{0.3cm}\vert\hspace{0.3cm}((1,7) ,11,$Δ$ ) \hspace{0.3cm}\vert\hspace{0.3cm} ((4,3) ,11.5,$Κ$ ) \hspace{0.3cm}\vert\hspace{0.3cm}\\ ((2,2) , 12 , $Δ$) \hspace{0.3cm}\vert\hspace{0.3cm} ((0,6) ,12,$Π$ ) $]
     \newline
     \vspace{-150cm}
     \vspace*{150cm}
     Explored: [$S , (1,0) , (0,1) , (2,0) , (0,2) , (3,0) , (0,3) ,(4,0) , (3,1) , (1,3) ,(1,4) , (1,5) , (2,1) ,\\ (4,1) , (3,2) , (2,3) ,(2,5) ,(1,6) ,(3,3)$]
    \newline
    \newline
    $\bullet$\hspace{0.1cm}Αφαιρειται ο $(3,5)$ (που ειναι η Κατω ενεργεια του ρομποτ απο τον κομβο $(2,5$) και εισαγονται οι γειτονες του $((4,5) ,11.5,$Κ$ ) , ((3,6) ,12.5,$Δ$ )$ , το $(2,5)$ δεν το ξαναεισαγουμε διοτι για να παμε σε αυτο μεσω του $(3,5)$ εχουμε $f = 8 + 4.5 =12.5 > 10.5$ 
    \newline
     \vspace{-150cm}
     \vspace*{150cm}Fringe: [ $((2,6) ,11,$Δ$ ) \hspace{0.3cm}\vert\hspace{0.3cm}((1,7) ,11,$Δ$ ) \hspace{0.3cm}\vert\hspace{0.3cm} ((4,3) ,11.5,$Κ$ ) \hspace{0.3cm}\vert\hspace{0.3cm}((4,5) ,11.5,$Κ$ )\hspace{0.3cm}\vert\hspace{0.3cm}\\ ((2,2) , 12 , $Δ$) \hspace{0.3cm}\vert\hspace{0.3cm} ((0,6) ,12,$Π$ )\hspace{0.3cm}\vert\hspace{0.3cm}((3,6) ,12.5,$Δ$ )$]
     \newline
     \vspace{-150cm}
     \vspace*{150cm}
     Explored: [$S , (1,0) , (0,1) , (2,0) , (0,2) , (3,0) , (0,3) ,(4,0) , (3,1) , (1,3) ,(1,4) , (1,5) , (2,1) ,\\ (4,1) , (3,2) , (2,3) ,(2,5) ,(1,6) ,(3,3) ,(3,5)$]
    \newline
    \newline
    $\bullet$\hspace{0.1cm}Αφαιρειται ο $(2,6)$ (που ειναι η Δεξια ενεργεια του ρομποτ απο τον κομβο $(2,5$) και εισαγονται οι γειτονες του $((2,7) ,11.5,$Δ$ )$ , το $(1,6)$ δεν το ξαναεισαγουμε διοτι για να παμε σε αυτο μεσω του $(2,6)$ εχουμε $f = 8 + 4.5 =12.5 > 10.5$ , ομοιως και για το $(2,5)$ . Το $(3,6)$ δεν το ανταλασουμε με αυτο του frontier διοτι εχει $f = 9 + 3.5 = 12.5$
    \newline
     \vspace{-150cm}
     \vspace*{150cm}Fringe: [ $((1,7) ,11,$Δ$ )\hspace{0.3cm}\vert\hspace{0.3cm} ((4,3) ,11.5,$Κ$ ) \hspace{0.3cm}\vert\hspace{0.3cm}((4,5) ,11.5,$Κ$ )\hspace{0.3cm}\vert\hspace{0.3cm}((2,7) ,11.5,$Δ$ )\hspace{0.3cm}\vert\hspace{0.3cm}\\ ((2,2) , 12 , $Δ$) \hspace{0.3cm}\vert\hspace{0.3cm} ((0,6) ,12,$Π$ )\hspace{0.3cm}\vert\hspace{0.3cm}((3,6) ,12.5,$Δ$ )$]
     \newline
     \vspace{-150cm}
     \vspace*{150cm}
     Explored: [$S , (1,0) , (0,1) , (2,0) , (0,2) , (3,0) , (0,3) ,(4,0) , (3,1) , (1,3) ,(1,4) , (1,5) , (2,1) ,\\ (4,1) , (3,2) , (2,3) ,(2,5) ,(1,6) ,(3,3) ,(3,5) ,(2,6)$]
    \newline
    \newline
    $\bullet$\hspace{0.1cm}Αφαιρειται ο $(1,7)$ (που ειναι η Δεξια ενεργεια του ρομποτ απο τον κομβο $(1,6$) και εισαγονται οι γειτονες του $((0,7) ,12,$Π$ )$ .Το $(2,7)$ δεν το ανταλασουμε με αυτο του frontier διοτι εχει $f = 8 + 3.5 = 11.5$ 
    \newline
     \vspace{-150cm}
     \vspace*{150cm}Fringe: [ $((4,3) ,11.5,$Κ$ ) \hspace{0.3cm}\vert\hspace{0.3cm}((4,5) ,11.5,$Κ$ )\hspace{0.3cm}\vert\hspace{0.3cm}((2,7) ,11.5,$Δ$ )\hspace{0.3cm}\vert\hspace{0.3cm} ((2,2) , 12 , $Δ$) \hspace{0.3cm}\vert\hspace{0.3cm}\\ ((0,6) ,12,$Π$ )\hspace{0.3cm}\vert\hspace{0.3cm}((0,7) ,12,$Π$ )\hspace{0.3cm}\vert\hspace{0.3cm}((3,6) ,12.5,$Δ$ )$]
     \newline
     \vspace{-150cm}
     \vspace*{150cm}
     Explored: [$S , (1,0) , (0,1) , (2,0) , (0,2) , (3,0) , (0,3) ,(4,0) , (3,1) , (1,3) ,(1,4) , (1,5) , (2,1) ,\\ (4,1) , (3,2) , (2,3) ,(2,5) ,(1,6) ,(3,3) ,(3,5) ,(2,6) , (1,7)$]
    \newline
    \newline
    $\bullet$\hspace{0.1cm}Αφαιρειται ο $(4,3)$ (που ειναι η Κατω ενεργεια του ρομποτ απο τον κομβο $(3,3$) και εισαγονται οι γειτονες του $((5,3) ,13,$Κ$ ) ,((4,4) , 12,$Δ$ ) $
    \newline
     \vspace{-150cm}
     \vspace*{150cm}Fringe: [ $ ((4,5) ,11.5,$Κ$ )\hspace{0.3cm}\vert\hspace{0.3cm}((2,7) ,11.5,$Δ$ )\hspace{0.3cm}\vert\hspace{0.3cm} ((2,2) , 12 , $Δ$) \hspace{0.3cm}\vert\hspace{0.3cm} ((0,6) ,12,$Π$ )\hspace{0.3cm}\vert\hspace{0.3cm}\\((0,7) ,12,$Π$ )\hspace{0.3cm}\vert\hspace{0.3cm} ((4,4) , 12,$Δ$ )\hspace{0.3cm}\vert\hspace{0.3cm}((3,6) ,12.5,$Δ$ ) \hspace{0.3cm}\vert\hspace{0.3cm}((5,3) ,13,$Κ$ )$]
     \newline
     \vspace{-150cm}
     \vspace*{150cm}
     Explored: [$S , (1,0) , (0,1) , (2,0) , (0,2) , (3,0) , (0,3) ,(4,0) , (3,1) , (1,3) ,(1,4) , (1,5) , (2,1) ,\\ (4,1) , (3,2) , (2,3) ,(2,5) ,(1,6) ,(3,3) ,(3,5) ,(2,6) , (1,7), (4,3)$]
    \newline
    \newline
    $\bullet$\hspace{0.1cm}Αφαιρειται ο $(4,5)$ (που ειναι η Κατω ενεργεια του ρομποτ απο τον κομβο $(3,5$) και εισαγονται οι γειτονες του $((5,5) ,12,$Κ$ ) ,((4,6) , 11.5,$Δ$ )$ , το $(4,4)$ που ειναι μεσα στο fringe δεν του κανουμε καποια αλλαγη καθως για να παμε μεσω του $(4,5)$ σε αυτο εχουμε $f = 9 + 4 = 13 > 12$
    \newline
     \vspace{-150cm}
     \vspace*{150cm}Fringe: [ $ ((2,7) ,11.5,$Δ$ )\hspace{0.3cm}\vert\hspace{0.3cm}((4,6) , 11.5,$Δ$ )\hspace{0.3cm}\vert\hspace{0.3cm} ((2,2) , 12 , $Δ$) \hspace{0.3cm}\vert\hspace{0.3cm} ((0,6) ,12,$Π$ )\hspace{0.3cm}\vert\hspace{0.3cm}\\((0,7) ,12,$Π$ )\hspace{0.3cm}\vert\hspace{0.3cm} ((4,4) , 12,$Δ$ )\hspace{0.3cm}\vert\hspace{0.3cm}((5,5) ,12,$Κ$ )  \hspace{0.3cm}\vert\hspace{0.3cm} ((3,6) ,12.5,$Δ$ ) \hspace{0.3cm}\vert\hspace{0.3cm}((5,3) ,13,$Κ$ )$]
     \newline
     \vspace{-150cm}
     \vspace*{150cm}
     Explored: [$S , (1,0) , (0,1) , (2,0) , (0,2) , (3,0) , (0,3) ,(4,0) , (3,1) , (1,3) ,(1,4) , (1,5) , (2,1) ,\\ (4,1) , (3,2) , (2,3) ,(2,5) ,(1,6) ,(3,3) ,(3,5) ,(2,6) , (1,7), (4,3), (4,5)$]
    \newline
    \newline
    $\bullet$\hspace{0.1cm}Αφαιρειται ο $(2,7)$ (που ειναι η Δεξια ενεργεια του ρομποτ απο τον κομβο $(2,6$) και εισαγονται οι γειτονες του $((3,7) ,13,$Κ$ )$ , το $(1,7)$ δεν το ξαναεισαγουμε στο fringe διοτι μεσω του $(2,7)$ θα εχουμε $f = 9 + 4 = 13 > 11$ , ομοιως και για το $(2,6)$
    \newline
     \vspace{-150cm}
     \vspace*{150cm}Fringe: [ $((4,6) , 11.5,$Δ$ )\hspace{0.3cm}\vert\hspace{0.3cm} ((2,2) , 12 , $Δ$) \hspace{0.3cm}\vert\hspace{0.3cm} ((0,6) ,12,$Π$ )\hspace{0.3cm}\vert\hspace{0.3cm}(0,7) ,12,$Π$ )\hspace{0.3cm}\vert\hspace{0.3cm}\\ ((4,4) , 12,$Δ$ )\hspace{0.3cm}\vert\hspace{0.3cm}((5,5) ,12,$Κ$ )  \hspace{0.3cm}\vert\hspace{0.3cm} ((3,6) ,12.5,$Δ$ ) \hspace{0.3cm}\vert\hspace{0.3cm}((5,3) ,13,$Κ$ )\hspace{0.3cm}\vert\hspace{0.3cm}((3,7) ,13,$Κ$ )$]
     \newline
     \vspace{-150cm}
     \vspace*{150cm}
     Explored: [$S , (1,0) , (0,1) , (2,0) , (0,2) , (3,0) , (0,3) ,(4,0) , (3,1) , (1,3) ,(1,4) , (1,5) , (2,1) ,\\ (4,1) , (3,2) , (2,3) ,(2,5) ,(1,6) ,(3,3) ,(3,5) ,(2,6) , (1,7), (4,3), (4,5) ,(2,7)$]
    \newline
    \newline
    $\bullet$\hspace{0.1cm}Αφαιρειται ο $(4,6)$ (που ειναι η Δεξια ενεργεια του ρομποτ απο τον κομβο $(4,5$) και εισαγονται οι γειτονες του $((4,7) ,11.5, $Δ$ )$ , το $(3,6)$ δεν το ανταλασουμε με αυτο που υπαρχει στο fringe διοτι μεσω του $(4,6)$ θα εχουμε $f = 10.5 + 3.5 = 14 > 12.5$ 
    \newline
     \vspace{-150cm}
     \vspace*{150cm}Fringe: [ $ ((4,7) ,11.5, $Δ$ )\hspace{0.3cm}\vert\hspace{0.3cm} ((2,2) , 12 , $Δ$) \hspace{0.3cm}\vert\hspace{0.3cm} ((0,6) ,12,$Π$ )\hspace{0.3cm}\vert\hspace{0.3cm}(0,7) ,12,$Π$ )\hspace{0.3cm}\vert\hspace{0.3cm}\\ ((4,4) , 12,$Δ$ )\hspace{0.3cm}\vert\hspace{0.3cm}((5,5) ,12,$Κ$ )  \hspace{0.3cm}\vert\hspace{0.3cm} ((3,6) ,12.5,$Δ$ ) \hspace{0.3cm}\vert\hspace{0.3cm}((5,3) ,13,$Κ$ )\hspace{0.3cm}\vert\hspace{0.3cm}((3,7) ,13,$Κ$ )$]
     \newline
     \vspace{-150cm}
     \vspace*{150cm}
     Explored: [$S , (1,0) , (0,1) , (2,0) , (0,2) , (3,0) , (0,3) ,(4,0) , (3,1) , (1,3) ,(1,4) , (1,5) , (2,1) ,\\ (4,1) , (3,2) , (2,3) ,(2,5) ,(1,6) ,(3,3) ,(3,5) ,(2,6) , (1,7), (4,3), (4,5) ,(2,7) ,(4,6)$]
    \newline
    \newline
    $\bullet$\hspace{0.1cm}Αφαιρειται ο $(4,7)$ (που ειναι η Δεξια ενεργεια του ρομποτ απο τον κομβο $(4,6$) και εισαγονται οι γειτονες του $((5,7) ,12, $Κ$ ) , ((4,8) ,13, $Δ$ )$ , το $(3,7)$ δεν το ανταλασουμε με αυτο που υπαρχει στο fringe διοτι μεσω του $(4,7)$ θα εχουμε $f = 11 + 3 = 14 > 13$ 
    \newline
     \vspace{-150cm}
     \vspace*{150cm}Fringe: [ $((2,2) , 12 , $Δ$) \hspace{0.3cm}\vert\hspace{0.3cm} ((0,6) ,12,$Π$ )\hspace{0.3cm}\vert\hspace{0.3cm}(0,7) ,12,$Π$ )\hspace{0.3cm}\vert\hspace{0.3cm}((4,4) , 12,$Δ$ )\hspace{0.3cm}\vert\hspace{0.3cm}\\((5,5) ,12,$Κ$ )  \hspace{0.3cm}\vert\hspace{0.3cm}((5,7) ,12, $Κ$ ) \hspace{0.3cm}\vert\hspace{0.3cm}((3,6) ,12.5,$Δ$ ) \hspace{0.3cm}\vert\hspace{0.3cm}((5,3) ,13,$Κ$ )\hspace{0.3cm}\vert\hspace{0.3cm}((3,7) ,13,$Κ$ ) \hspace{0.3cm}\vert\hspace{0.3cm}\\ ((4,8) ,13, $Δ$ )$]
     \newline
     \vspace{-150cm}
     \vspace*{150cm}
     Explored: [$S , (1,0) , (0,1) , (2,0) , (0,2) , (3,0) , (0,3) ,(4,0) , (3,1) , (1,3) ,(1,4) , (1,5) , (2,1) ,\\ (4,1) , (3,2) , (2,3) ,(2,5) ,(1,6) ,(3,3) ,(3,5) ,(2,6) , (1,7), (4,3), (4,5) ,(2,7) ,(4,6) ,(4,7)$]
    \newline
    \newline
    $\bullet$\hspace{0.1cm}Αφαιρειται ο $(2,2)$ (που ειναι η Δεξια ενεργεια του ρομποτ απο τον κομβο $(2,1$) , δεν εισαγουμε καποιον γειτονα διοτι για να παμε στο $(3,2)$ (που το εχουμε κανει visit) μεσω του $(2,2)$ θα εχουμε $f = 7 + 5.5 = 12.5 > 10.5$ ομοιως και για το $(2,3)$ εχουμε $f = 7 + 6 = 13 > 10.5$
    \newline
     \vspace{-150cm}
     \vspace*{150cm}Fringe: [ $((0,6) ,12,$Π$ )\hspace{0.3cm}\vert\hspace{0.3cm}(0,7) ,12,$Π$ )\hspace{0.3cm}\vert\hspace{0.3cm}((4,4) , 12,$Δ$ )\hspace{0.3cm}\vert\hspace{0.3cm}((5,5) ,12,$Κ$ )  \hspace{0.3cm}\vert\hspace{0.3cm}\\((5,7) ,12, $Κ$ ) \hspace{0.3cm}\vert\hspace{0.3cm}((3,6) ,12.5,$Δ$ ) \hspace{0.3cm}\vert\hspace{0.3cm}((5,3) ,13,$Κ$ )\hspace{0.3cm}\vert\hspace{0.3cm}((3,7) ,13,$Κ$ ) \hspace{0.3cm}\vert\hspace{0.3cm}\\ ((4,8) ,13, $Δ$ )$]
     \newline
     \vspace{-150cm}
     \vspace*{150cm}
     Explored: [$S , (1,0) , (0,1) , (2,0) , (0,2) , (3,0) , (0,3) ,(4,0) , (3,1) , (1,3) ,(1,4) , (1,5) , (2,1) ,\\ (4,1) , (3,2) , (2,3) ,(2,5) ,(1,6) ,(3,3) ,(3,5) ,(2,6) , (1,7), (4,3), (4,5) ,(2,7) ,(4,6) ,(4,7) ,\\(2,2)$]
    \newline
    \newline
    $\bullet$\hspace{0.1cm}Αφαιρειται ο $(0,6)$ (που ειναι η Πανω ενεργεια του ρομποτ απο τον κομβο $(1,6$) , δεν εχουμε να εισαγουμε καποιον γειτονα  ,το $(0,7)$ που ειναι μεσα στο frontier εαν το κανουμε visit μεσω του $(0,6)$ θα εχουμε $f = 7.5 + 4.5 = 12 = 12$ του frontier
    \newline
     \vspace{-150cm}
     \vspace*{150cm}Fringe: [ $((0,7) ,12,$Π$ )\hspace{0.3cm}\vert\hspace{0.3cm}((4,4) , 12,$Δ$ )\hspace{0.3cm}\vert\hspace{0.3cm}((5,5) ,12,$Κ$ )  \hspace{0.3cm}\vert\hspace{0.3cm}((5,7) ,12, $Κ$ ) \hspace{0.3cm}\vert\hspace{0.3cm}\\((3,6) ,12.5,$Δ$ ) \hspace{0.3cm}\vert\hspace{0.3cm}((5,3) ,13,$Κ$ )\hspace{0.3cm}\vert\hspace{0.3cm}((3,7) ,13,$Κ$ ) \hspace{0.3cm}\vert\hspace{0.3cm}((4,8) ,13, $Δ$ )$]
     \newline
     \vspace{-150cm}
     \vspace*{150cm}
     Explored: [$S , (1,0) , (0,1) , (2,0) , (0,2) , (3,0) , (0,3) ,(4,0) , (3,1) , (1,3) ,(1,4) , (1,5) , (2,1) ,\\ (4,1) , (3,2) , (2,3) ,(2,5) ,(1,6) ,(3,3) ,(3,5) ,(2,6) , (1,7), (4,3), (4,5) ,(2,7) ,(4,6) ,(4,7) ,\\(2,2) ,(0,6)$]
    \newline
    \newline
    $\bullet$\hspace{0.1cm}Αφαιρειται ο $(0,7)$ (που ειναι η Πανω ενεργεια του ρομποτ απο τον κομβο $(1,7$) ) και εισαγονται οι γειτονες του $((0,8) ,13, $Δ$ )$ , δεν εισαγουμε παλι στο frontier το $(0,6)$ διοτι για να παμε απο το $(0,7)$ σε αυτο θα εχουμε κοστος $f = 8.5 + 5 = 13.5 > 12$ που ηταν το κοστος f του $(0,6)$ 
    \newline
     \vspace{-150cm}
     \vspace*{150cm}Fringe: [ $((4,4) , 12,$Δ$ )\hspace{0.3cm}\vert\hspace{0.3cm}((5,5) ,12,$Κ$ )  \hspace{0.3cm}\vert\hspace{0.3cm}((5,7) ,12, $Κ$ ) \hspace{0.3cm}\vert\hspace{0.3cm}((3,6) ,12.5,$Δ$ ) \hspace{0.3cm}\vert\hspace{0.3cm}\\((5,3) ,13,$Κ$ )\hspace{0.3cm}\vert\hspace{0.3cm}((3,7) ,13,$Κ$ ) \hspace{0.3cm}\vert\hspace{0.3cm}((4,8) ,13, $Δ$ )\hspace{0.3cm}\vert\hspace{0.3cm}((0,8) ,13, $Δ$ ) $]
     \newline
     \vspace{-150cm}
     \vspace*{150cm}
     Explored: [$S , (1,0) , (0,1) , (2,0) , (0,2) , (3,0) , (0,3) ,(4,0) , (3,1) , (1,3) ,(1,4) , (1,5) , (2,1) ,\\ (4,1) , (3,2) , (2,3) ,(2,5) ,(1,6) ,(3,3) ,(3,5) ,(2,6) , (1,7), (4,3), (4,5) ,(2,7) ,(4,6) ,(4,7) ,\\(2,2) ,(0,6) ,(0,7)$]
    \newline
    \newline
    $\bullet$\hspace{0.1cm}Αφαιρειται ο $(4,4)$ (που ειναι η Δεξια ενεργεια του ρομποτ απο τον κομβο $(4,3$) ) και εισαγονται οι γειτονες του $((5,4) ,13.5, $Κ$ )$ , δεν εισαγουμε παλι στο frontier το $(4,5)$ διοτι για να παμε απο το $(4,4)$ σε αυτο θα εχουμε κοστος $f = 9 + 3.5 = 12.5 > 11.5$ που ηταν το κοστος f του $(4,5)$ 
    \newline
     \vspace{-150cm}
     \vspace*{150cm}Fringe: [ $((5,5) ,12,$Κ$ )  \hspace{0.3cm}\vert\hspace{0.3cm}((5,7) ,12, $Κ$ ) \hspace{0.3cm}\vert\hspace{0.3cm}((3,6) ,12.5,$Δ$ ) \hspace{0.3cm}\vert\hspace{0.3cm}((5,3) ,13,$Κ$ )\hspace{0.3cm}\vert\hspace{0.3cm}\\((3,7) ,13,$Κ$ ) \hspace{0.3cm}\vert\hspace{0.3cm}((4,8) ,13, $Δ$ )\hspace{0.3cm}\vert\hspace{0.3cm}((0,8) ,13, $Δ$ )\hspace{0.3cm}\vert\hspace{0.3cm}((5,4) ,13.5, $Κ$ )  $]
     \newline
     \vspace{-150cm}
     \vspace*{150cm}
     Explored: [$S , (1,0) , (0,1) , (2,0) , (0,2) , (3,0) , (0,3) ,(4,0) , (3,1) , (1,3) ,(1,4) , (1,5) , (2,1) ,\\ (4,1) , (3,2) , (2,3) ,(2,5) ,(1,6) ,(3,3) ,(3,5) ,(2,6) , (1,7), (4,3), (4,5) ,(2,7) ,(4,6) ,(4,7) ,\\(2,2) ,(0,6) ,(0,7), (4,4)$]
    \newline
    \newline
    $\bullet$\hspace{0.1cm}Αφαιρειται ο $(5,5)$ (που ειναι η Κατω ενεργεια του ρομποτ απο τον κομβο $(4,5$) ) και εισαγονται οι γειτονες του $((6,5) ,12.5, $Κ$ )$ , δεν ανταλασουμε το $(5,4)$ που βρισκεται στο frontier διοτι για να παμε απο το $(5,5)$ σε αυτο θα εχουμε κοστος $f = 11 + 3.5 = 14.5 > 13.5$ που ειναι το κοστος f του $(5,4)$ στο frontier 
    \newline
     \vspace{-150cm}
     \vspace*{150cm}Fringe: [ $((5,7) ,12, $Κ$ ) \hspace{0.3cm}\vert\hspace{0.3cm}((3,6) ,12.5,$Δ$ ) \hspace{0.3cm}\vert\hspace{0.3cm}((6,5) ,12.5, $Κ$ )\hspace{0.3cm}\vert\hspace{0.3cm} ((5,3) ,13,$Κ$ )\hspace{0.3cm}\vert\hspace{0.3cm}\\((3,7) ,13,$Κ$ ) \hspace{0.3cm}\vert\hspace{0.3cm}((4,8) ,13, $Δ$ )\hspace{0.3cm}\vert\hspace{0.3cm}((0,8) ,13, $Δ$ )\hspace{0.3cm}\vert\hspace{0.3cm}((5,4) ,13.5, $Κ$ )  $]
     \newline
     \vspace{-150cm}
     \vspace*{150cm}
     Explored: [$S , (1,0) , (0,1) , (2,0) , (0,2) , (3,0) , (0,3) ,(4,0) , (3,1) , (1,3) ,(1,4) , (1,5) , (2,1) ,\\ (4,1) , (3,2) , (2,3) ,(2,5) ,(1,6) ,(3,3) ,(3,5) ,(2,6) , (1,7), (4,3), (4,5) ,(2,7) ,(4,6) ,(4,7) ,\\(2,2) ,(0,6) ,(0,7), (4,4) ,(5,5)$]
    \newline
    \newline
    $\bullet$\hspace{0.1cm}Αφαιρειται ο $(5,7)$ (που ειναι η Κατω ενεργεια του ρομποτ απο τον κομβο $(4,7$) ) και εισαγονται οι γειτονες του $((6,7) ,12.5, $Κ$ ) ,  ((5,8) ,13.5, $Δ$ )$ 
    \newline
     \vspace{-150cm}
     \vspace*{150cm}Fringe: [ $((3,6) ,12.5,$Δ$ ) \hspace{0.3cm}\vert\hspace{0.3cm}((6,5) ,12.5, $Κ$ )\hspace{0.3cm}\vert\hspace{0.3cm}((6,7) ,12.5, $Κ$ )\hspace{0.3cm}\vert\hspace{0.3cm}  ((5,3) ,13,$Κ$ )\hspace{0.3cm}\vert\hspace{0.3cm}\\((3,7) ,13,$Κ$ ) \hspace{0.3cm}\vert\hspace{0.3cm}((4,8) ,13, $Δ$ )\hspace{0.3cm}\vert\hspace{0.3cm}((0,8) ,13, $Δ$ )\hspace{0.3cm}\vert\hspace{0.3cm}((5,4) ,13.5, $Κ$ )\hspace{0.3cm}\vert\hspace{0.3cm} ((5,8) ,13.5, $Δ$ )  $ ]
     \newline
     \vspace{-150cm}
     \vspace*{150cm}
     Explored: [$S , (1,0) , (0,1) , (2,0) , (0,2) , (3,0) , (0,3) ,(4,0) , (3,1) , (1,3) ,(1,4) , (1,5) , (2,1) ,\\ (4,1) , (3,2) , (2,3) ,(2,5) ,(1,6) ,(3,3) ,(3,5) ,(2,6) , (1,7), (4,3), (4,5) ,(2,7) ,(4,6) ,(4,7) ,\\(2,2) ,(0,6) ,(0,7), (4,4) ,(5,5) ,(5,7)$]
    \newline
    \newline
    $\bullet$\hspace{0.1cm}Αφαιρειται ο $(3,6)$ (που ειναι η Δεξια ενεργεια του ρομποτ απο τον κομβο $(3,5$) ) και εισαγονται οι γειτονες του , το $(4,6)$ και το $(2,6)$ δεν τα ξαναεισαγουμε στο frontier καθως για να παμε σε αυτα μεσω του $(3,6)$ θα εχουμε μεγαλυτερο f απο αυτο που ειχαν, δηλαδη για το $(4,6)$ θα εχουμε $f = 10 + 3 = 13 >11.5$ που ηταν το κοστος f του $(4,6)$ , ομοιως και για το $(2,6)$ θα εχουμε $f = 9.5 + 4 = 13.5 > 11$ . Επισης δεν ανταλασουμε το $(3,7)$ που υπαρχει στο frontier διοτι εχουμε μεγαλυτερο f για να φτασουμε σε αυτο πηγενοντας μεσω του $(3,6)$ $f = 11 + 3 = 14 > 13$ 
    \newline
     \vspace{-150cm}
     \vspace*{150cm}Fringe: [ $((6,5) ,12.5, $Κ$ )\hspace{0.3cm}\vert\hspace{0.3cm}((6,7) ,12.5, $Κ$ )\hspace{0.3cm}\vert\hspace{0.3cm}  ((5,3) ,13,$Κ$ )\hspace{0.3cm}\vert\hspace{0.3cm}((3,7) ,13,$Κ$ ) \hspace{0.3cm}\vert\hspace{0.3cm}\\((4,8) ,13, $Δ$ )\hspace{0.3cm}\vert\hspace{0.3cm}((0,8) ,13, $Δ$ )\hspace{0.3cm}\vert\hspace{0.3cm}((5,4) ,13.5, $Κ$ )\hspace{0.3cm}\vert\hspace{0.3cm} ((5,8) ,13.5, $Δ$ )  $ ]
     \newline
     \vspace{-150cm}
     \vspace*{150cm}
     Explored: [$S , (1,0) , (0,1) , (2,0) , (0,2) , (3,0) , (0,3) ,(4,0) , (3,1) , (1,3) ,(1,4) , (1,5) , (2,1) ,\\ (4,1) , (3,2) , (2,3) ,(2,5) ,(1,6) ,(3,3) ,(3,5) ,(2,6) , (1,7), (4,3), (4,5) ,(2,7) ,(4,6) ,(4,7) ,\\(2,2) ,(0,6) ,(0,7), (4,4) ,(5,5) ,(5,7) ,(3,6)$]
    \newline
    \newline
    $\bullet$\hspace{0.1cm}Αφαιρειται ο $(6,5)$ (που ειναι η Κατω ενεργεια του ρομποτ απο τον κομβο $(5,5$) ) και εισαγονται οι γειτονες του $((6,4) ,14, $Α$ ) ,  ((6,6) ,13, $Δ$ )$ 
    \newline
     \vspace{-150cm}
     \vspace*{150cm}Fringe: [ $((6,7) ,12.5, $Κ$ )\hspace{0.3cm}\vert\hspace{0.3cm}  ((5,3) ,13,$Κ$ )\hspace{0.3cm}\vert\hspace{0.3cm}((3,7) ,13,$Κ$ ) \hspace{0.3cm}\vert\hspace{0.3cm}((4,8) ,13, $Δ$ )\hspace{0.3cm}\vert\hspace{0.3cm}\\((0,8) ,13, $Δ$ )\hspace{0.3cm}\vert\hspace{0.3cm} ((6,6) ,13, $Δ$ ) \hspace{0.3cm}\vert\hspace{0.3cm} ((5,4) ,13.5, $Κ$ )\hspace{0.3cm}\vert\hspace{0.3cm} ((5,8) ,13.5, $Δ$ )\hspace{0.3cm}\vert\hspace{0.3cm} ((6,4) ,14, $Α$ )  $ ]
     \newline
     \vspace{-150cm}
     \vspace*{150cm}
     Explored: [$S , (1,0) , (0,1) , (2,0) , (0,2) , (3,0) , (0,3) ,(4,0) , (3,1) , (1,3) ,(1,4) , (1,5) , (2,1) ,\\ (4,1) , (3,2) , (2,3) ,(2,5) ,(1,6) ,(3,3) ,(3,5) ,(2,6) , (1,7), (4,3), (4,5) ,(2,7) ,(4,6) ,(4,7) ,\\(2,2) ,(0,6) ,(0,7), (4,4) ,(5,5) ,(5,7) ,(3,6) ,(6,5)$]
    \newline
    \newline
    $\bullet$\hspace{0.1cm}Αφαιρειται ο $(6,7)$ (που ειναι η Κατω ενεργεια του ρομποτ απο τον κομβο $(5,7$) ) και εισαγονται οι γειτονες του $((7,7) ,12.5, $Κ$ )$ το $(6,6)$ δεν το ανταλασουμε με αυτο του frontier διοτι εχουμε μεγαλυτερο f cost μεσω του $(6,7)$ $f = 12+2 = 14> 13$
    \newline
     \vspace{-150cm}
     \vspace*{150cm}Fringe: [ $((7,7) ,12.5, $Κ$ )\hspace{0.3cm}\vert\hspace{0.3cm}((5,3) ,13,$Κ$ )\hspace{0.3cm}\vert\hspace{0.3cm}((3,7) ,13,$Κ$ ) \hspace{0.3cm}\vert\hspace{0.3cm}((4,8) ,13, $Δ$ )\hspace{0.3cm}\vert\hspace{0.3cm}\\((0,8) ,13, $Δ$ )\hspace{0.3cm}\vert\hspace{0.3cm} ((6,6) ,13, $Δ$ ) \hspace{0.3cm}\vert\hspace{0.3cm} ((5,4) ,13.5, $Κ$ )\hspace{0.3cm}\vert\hspace{0.3cm} ((5,8) ,13.5, $Δ$ )\hspace{0.3cm}\vert\hspace{0.3cm} ((6,4) ,14, $Α$ )  $ ]
     \newline
     \vspace{-150cm}
     \vspace*{150cm}
     Explored: [$S , (1,0) , (0,1) , (2,0) , (0,2) , (3,0) , (0,3) ,(4,0) , (3,1) , (1,3) ,(1,4) , (1,5) , (2,1) ,\\ (4,1) , (3,2) , (2,3) ,(2,5) ,(1,6) ,(3,3) ,(3,5) ,(2,6) , (1,7), (4,3), (4,5) ,(2,7) ,(4,6) ,(4,7) ,\\(2,2) ,(0,6) ,(0,7), (4,4) ,(5,5) ,(5,7) ,(3,6) ,(6,5) ,(6,7)$]
     \newline
     \newline
    $\bullet$\hspace{0.1cm}Αφαιρειται ο $(7,7)$ (που ειναι η Κατω ενεργεια του ρομποτ απο τον κομβο $(6,7$) ) και εισαγονται οι γειτονες του $((8,7) ,13, $Κ$ ) ,((7,6) ,13.5, $Α$ ) , ((7,8) ,14, $Δ$ )$ 
    \newline
     \vspace{-150cm}
     \vspace*{150cm}Fringe: [ $((5,3) ,13,$Κ$ )\hspace{0.3cm}\vert\hspace{0.3cm}((3,7) ,13,$Κ$ ) \hspace{0.3cm}\vert\hspace{0.3cm}((4,8) ,13, $Δ$ )\hspace{0.3cm}\vert\hspace{0.3cm}((0,8) ,13, $Δ$ )\hspace{0.3cm}\vert\hspace{0.3cm} ((6,6) ,13, $Δ$ ) \hspace{0.3cm}\vert\hspace{0.3cm}\\ ((8,7) ,13, $Κ$ )\hspace{0.3cm}\vert\hspace{0.3cm} ((5,4) ,13.5, $Κ$ )\hspace{0.3cm}\vert\hspace{0.3cm} ((5,8) ,13.5, $Δ$ )\hspace{0.3cm}\vert\hspace{0.3cm} ((7,6) ,13.5, $Α$ )\hspace{0.3cm}\vert\hspace{0.3cm}  ((6,4) ,14, $Α$ ) \hspace{0.3cm}\vert\hspace{0.3cm}\\  ((7,8) ,14, $Δ$ ) $ ]
     \newline
     \vspace{-150cm}
     \vspace*{150cm}
     \newpage
     Explored: [$S , (1,0) , (0,1) , (2,0) , (0,2) , (3,0) , (0,3) ,(4,0) , (3,1) , (1,3) ,(1,4) , (1,5) , (2,1) ,\\ (4,1) , (3,2) , (2,3) ,(2,5) ,(1,6) ,(3,3) ,(3,5) ,(2,6) , (1,7), (4,3), (4,5) ,(2,7) ,(4,6) ,(4,7) ,\\(2,2) ,(0,6) ,(0,7), (4,4) ,(5,5) ,(5,7) ,(3,6) ,(6,5) ,(6,7) ,(7,7)$]
     \newline
     \newline
     $\bullet$\hspace{0.1cm}Αφαιρειται ο $(5,3)$ (που ειναι η Κατω ενεργεια του ρομποτ απο τον κομβο $(4,3$) ) και εισαγονται οι γειτονες του $((6,3) ,13.5, $Κ$ ) ,((5,2) ,14.5, $Α$ ) $ , δεν κανουμε update το $(5,4)$  μεσα στο frontier γιατι μεσω του $(5,3)$ εχουμε $f = 11 + 3.5 = 14.5 >13.5$ 
    \newline
     \vspace{-150cm}
     \vspace*{150cm}Fringe: [ $((3,7) ,13,$Κ$ ) \hspace{0.3cm}\vert\hspace{0.3cm}((4,8) ,13, $Δ$ )\hspace{0.3cm}\vert\hspace{0.3cm}((0,8) ,13, $Δ$ )\hspace{0.3cm}\vert\hspace{0.3cm} ((6,6) ,13, $Δ$ ) \hspace{0.3cm}\vert\hspace{0.3cm} ((8,7) ,13, $Κ$ )\hspace{0.3cm}\vert\hspace{0.3cm}\\ ((5,4) ,13.5, $Κ$ )\hspace{0.3cm}\vert\hspace{0.3cm} ((5,8) ,13.5, $Δ$ )\hspace{0.3cm}\vert\hspace{0.3cm} ((7,6) ,13.5, $Α$ )\hspace{0.3cm}\vert\hspace{0.3cm} ((6,3) ,13.5, $Κ$ )\hspace{0.3cm}\vert\hspace{0.3cm} ((6,4) ,14, $Α$ ) \hspace{0.3cm}\vert\hspace{0.3cm}\\  ((7,8) ,14, $Δ$ ) \hspace{0.3cm}\vert\hspace{0.3cm} ((5,2) ,14.5, $Α$ ) $ ]
     \newline
     \vspace{-150cm}
     \vspace*{150cm}
     Explored: [$S , (1,0) , (0,1) , (2,0) , (0,2) , (3,0) , (0,3) ,(4,0) , (3,1) , (1,3) ,(1,4) , (1,5) , (2,1) ,\\ (4,1) , (3,2) , (2,3) ,(2,5) ,(1,6) ,(3,3) ,(3,5) ,(2,6) , (1,7), (4,3), (4,5) ,(2,7) ,(4,6) ,(4,7) ,\\(2,2) ,(0,6) ,(0,7), (4,4) ,(5,5) ,(5,7) ,(3,6) ,(6,5) ,(6,7) ,(7,7) , (5,3)$]
     \newline
     \newline
     $\bullet$\hspace{0.1cm}Αφαιρειται ο $(3,7)$ (που ειναι η Κατω ενεργεια του ρομποτ απο τον κομβο $(2,7$) ) και εισαγονται οι γειτονες του $((3,8) ,14.5, $Δ$ ) $ 
    \newline
     \vspace{-150cm}
     \vspace*{150cm}Fringe: [ $((4,8) ,13, $Δ$ )\hspace{0.3cm}\vert\hspace{0.3cm}((0,8) ,13, $Δ$ )\hspace{0.3cm}\vert\hspace{0.3cm} ((6,6) ,13, $Δ$ ) \hspace{0.3cm}\vert\hspace{0.3cm} ((8,7) ,13, $Κ$ )\hspace{0.3cm}\vert\hspace{0.3cm}((5,4) ,13.5, $Κ$ )\hspace{0.3cm}\vert\hspace{0.3cm}\\ ((5,8) ,13.5, $Δ$ )\hspace{0.3cm}\vert\hspace{0.3cm} ((7,6) ,13.5, $Α$ )\hspace{0.3cm}\vert\hspace{0.3cm} ((6,3) ,13.5, $Κ$ )\hspace{0.3cm}\vert\hspace{0.3cm} ((6,4) ,14, $Α$ ) \hspace{0.3cm}\vert\hspace{0.3cm}((7,8) ,14, $Δ$ ) \hspace{0.3cm}\vert\hspace{0.3cm}\\ ((5,2) ,14.5, $Α$ ) \hspace{0.3cm}\vert\hspace{0.3cm} ((3,8) ,14.5, $Δ$ )$ ]
     \newline
     \vspace{-150cm}
     \vspace*{150cm}
     Explored: [$S , (1,0) , (0,1) , (2,0) , (0,2) , (3,0) , (0,3) ,(4,0) , (3,1) , (1,3) ,(1,4) , (1,5) , (2,1) ,\\ (4,1) , (3,2) , (2,3) ,(2,5) ,(1,6) ,(3,3) ,(3,5) ,(2,6) , (1,7), (4,3), (4,5) ,(2,7) ,(4,6) ,(4,7) ,\\(2,2) ,(0,6) ,(0,7), (4,4) ,(5,5) ,(5,7) ,(3,6) ,(6,5) ,(6,7) ,(7,7) , (5,3) ,(3,7)$]
     \newline
     \newline
     $\bullet$\hspace{0.1cm}Αφαιρειται ο $(4,8)$ (που ειναι η Δεξια ενεργεια του ρομποτ απο τον κομβο $(4,7$) ) και εισαγονται οι γειτονες του . Δεν κανουμε update ουτε το $(3,8) , f = 11 + 3.5 = 14.5$ ουτε το $(5,8) , f = 11 + 2.5 = 13.5$ , αρα δεν εχουμε καποιον γειτονα να εισαγουμε.
    \newline
     \vspace{-150cm}
     \vspace*{150cm}Fringe: [ $((0,8) ,13, $Δ$ )\hspace{0.3cm}\vert\hspace{0.3cm} ((6,6) ,13, $Δ$ ) \hspace{0.3cm}\vert\hspace{0.3cm} ((8,7) ,13, $Κ$ )\hspace{0.3cm}\vert\hspace{0.3cm}((5,4) ,13.5, $Κ$ )\hspace{0.3cm}\vert\hspace{0.3cm} ((5,8) ,13.5, $Δ$ )\hspace{0.3cm}\vert\hspace{0.3cm}\\ ((7,6) ,13.5, $Α$ )\hspace{0.3cm}\vert\hspace{0.3cm} ((6,3) ,13.5, $Κ$ )\hspace{0.3cm}\vert\hspace{0.3cm} ((6,4) ,14, $Α$ ) \hspace{0.3cm}\vert\hspace{0.3cm}((7,8) ,14, $Δ$ ) \hspace{0.3cm}\vert\hspace{0.3cm} ((5,2) ,14.5, $Α$ ) \hspace{0.3cm}\vert\hspace{0.3cm}\\ ((3,8) ,14.5, $Δ$ )$ ]
     \newline
     \vspace{-150cm}
     \vspace*{150cm}
     Explored: [$S , (1,0) , (0,1) , (2,0) , (0,2) , (3,0) , (0,3) ,(4,0) , (3,1) , (1,3) ,(1,4) , (1,5) , (2,1) ,\\ (4,1) , (3,2) , (2,3) ,(2,5) ,(1,6) ,(3,3) ,(3,5) ,(2,6) , (1,7), (4,3), (4,5) ,(2,7) ,(4,6) ,(4,7) ,\\(2,2) ,(0,6) ,(0,7), (4,4) ,(5,5) ,(5,7) ,(3,6) ,(6,5) ,(6,7) ,(7,7) , (5,3) ,(3,7) ,(4,8)$]
     \newline
     \newline
    $\bullet$\hspace{0.1cm}Αφαιρειται ο $(0,8)$ (που ειναι η Δεξια ενεργεια του ρομποτ απο τον κομβο $(0,7$) ) και εισαγονται οι γειτονες του $((0,9) ,14 , $Δ$)$
    \newline
     \vspace{-150cm}
     \vspace*{150cm}Fringe: [ $((6,6) ,13, $Δ$ ) \hspace{0.3cm}\vert\hspace{0.3cm} ((8,7) ,13, $Κ$ )\hspace{0.3cm}\vert\hspace{0.3cm}((5,4) ,13.5, $Κ$ )\hspace{0.3cm}\vert\hspace{0.3cm} ((5,8) ,13.5, $Δ$ )\hspace{0.3cm}\vert\hspace{0.3cm}\\ ((7,6) ,13.5, $Α$ )\hspace{0.3cm}\vert\hspace{0.3cm} ((6,3) ,13.5, $Κ$ )\hspace{0.3cm}\vert\hspace{0.3cm} ((6,4) ,14, $Α$ ) \hspace{0.3cm}\vert\hspace{0.3cm}((7,8) ,14, $Δ$ ) \hspace{0.3cm}\vert\hspace{0.3cm}((0,9) ,14 , $Δ$)\hspace{0.3cm}\vert\hspace{0.3cm}\\ ((5,2) ,14.5, $Α$ ) \hspace{0.3cm}\vert\hspace{0.3cm} ((3,8) ,14.5, $Δ$ )$ ]
     \newline
     \vspace{-150cm}
     \vspace*{150cm}
     Explored: [$S , (1,0) , (0,1) , (2,0) , (0,2) , (3,0) , (0,3) ,(4,0) , (3,1) , (1,3) ,(1,4) , (1,5) , (2,1) ,\\ (4,1) , (3,2) , (2,3) ,(2,5) ,(1,6) ,(3,3) ,(3,5) ,(2,6) , (1,7), (4,3), (4,5) ,(2,7) ,(4,6) ,(4,7) ,\\(2,2) ,(0,6) ,(0,7), (4,4) ,(5,5) ,(5,7) ,(3,6) ,(6,5) ,(6,7) ,(7,7) , (5,3) ,(3,7) ,(4,8) ,(0,8)$]
     \newline
     \newline
     $\bullet$\hspace{0.1cm}Αφαιρειται ο $(6,6)$ (που ειναι η Δεξια ενεργεια του ρομποτ απο τον κομβο $(6,5$) ) και εισαγονται οι γειτονες του , κανουμε update τον κομβο $(7,6)$ που βρισκεται μεσα στο frontier καθως βρικαμε συντομοτερο path προς αυτον μεσω του $(6,6)$ με $f = 11.5 + 1.5 = 13 < 13.5$ ,επισης δεν ξαναεισαγουμε το $(6,7)$ γιατι μεσω του $(6,6)$ εχουμε $f = 12 + 1.5 = 13.5 > 12.5$
    \newline
     \vspace{-150cm}
     \vspace*{150cm}Fringe: [ $(8,7) ,13, $Κ$ )\hspace{0.3cm}\vert\hspace{0.3cm} ((7,6) ,13, $Κ$ ) \hspace{0.3cm}\vert\hspace{0.3cm} ((5,4) ,13.5, $Κ$ )\hspace{0.3cm}\vert\hspace{0.3cm} ((5,8) ,13.5, $Δ$ )\hspace{0.3cm}\vert\hspace{0.3cm}((6,3) ,13.5, $Κ$ )\hspace{0.3cm}\vert\hspace{0.3cm}\\ ((6,4) ,14, $Α$ ) \hspace{0.3cm}\vert\hspace{0.3cm}((7,8) ,14, $Δ$ ) \hspace{0.3cm}\vert\hspace{0.3cm}((0,9) ,14 , $Δ$)\hspace{0.3cm}\vert\hspace{0.3cm}((5,2) ,14.5, $Α$ ) \hspace{0.3cm}\vert\hspace{0.3cm} ((3,8) ,14.5, $Δ$ )$ ]
     \newline
     \vspace{-150cm}
     \vspace*{150cm}
     Explored: [$S , (1,0) , (0,1) , (2,0) , (0,2) , (3,0) , (0,3) ,(4,0) , (3,1) , (1,3) ,(1,4) , (1,5) , (2,1) ,\\ (4,1) , (3,2) , (2,3) ,(2,5) ,(1,6) ,(3,3) ,(3,5) ,(2,6) , (1,7), (4,3), (4,5) ,(2,7) ,(4,6) ,(4,7) ,\\(2,2) ,(0,6) ,(0,7), (4,4) ,(5,5) ,(5,7) ,(3,6) ,(6,5) ,(6,7) ,(7,7) , (5,3) ,(3,7) ,(4,8) ,(0,8) ,\\ (6,6)$]
     \newline
     \newline
    $\bullet$\hspace{0.1cm}Αφαιρειται ο $(8,7)$ (που ειναι η Κατω ενεργεια του ρομποτ απο τον κομβο $(7,7$) ) και εισαγονται οι γειτονες του $((9,7) , 13.5 , $Κ$) , ((8,6) , 14.5 , $Α$) $
    \newline
     \vspace{-150cm}
     \vspace*{150cm}Fringe: [ $((7,6) ,13, $Κ$ ) \hspace{0.3cm}\vert\hspace{0.3cm} ((5,4) ,13.5, $Κ$ )\hspace{0.3cm}\vert\hspace{0.3cm} ((5,8) ,13.5, $Δ$ )\hspace{0.3cm}\vert\hspace{0.3cm}((6,3) ,13.5, $Κ$ )\hspace{0.3cm}\vert\hspace{0.3cm}\\ ((9,7) , 13.5 , $Κ$) \hspace{0.3cm}\vert\hspace{0.3cm}  ((6,4) ,14, $Α$ ) \hspace{0.3cm}\vert\hspace{0.3cm}((7,8) ,14, $Δ$ ) \hspace{0.3cm}\vert\hspace{0.3cm}((0,9) ,14 , $Δ$)\hspace{0.3cm}\vert\hspace{0.3cm}((5,2) ,14.5, $Α$ ) \hspace{0.3cm}\vert\hspace{0.3cm}\\ ((3,8) ,14.5, $Δ$ ) \hspace{0.3cm}\vert\hspace{0.3cm}((8,6) , 14.5 , $Α$)$ ]
     \newline
     \vspace{-150cm}
     \vspace*{150cm}
     Explored: [$S , (1,0) , (0,1) , (2,0) , (0,2) , (3,0) , (0,3) ,(4,0) , (3,1) , (1,3) ,(1,4) , (1,5) , (2,1) ,\\ (4,1) , (3,2) , (2,3) ,(2,5) ,(1,6) ,(3,3) ,(3,5) ,(2,6) , (1,7), (4,3), (4,5) ,(2,7) ,(4,6) ,(4,7) ,\\(2,2) ,(0,6) ,(0,7), (4,4) ,(5,5) ,(5,7) ,(3,6) ,(6,5) ,(6,7) ,(7,7) , (5,3) ,(3,7) ,(4,8) ,(0,8) ,\\ (6,6) ,(8,7)$]
     \newline
     \newline
    $\bullet$\hspace{0.1cm}Αφαιρειται ο $(7,6)$ (που ειναι η Κατω ενεργεια του ρομποτ απο τον κομβο $(6,6$) ) και εισαγονται οι γειτονες του , κανουμε update το $(8,6)$ που ειναι μεσα στο fringe γιατι μεσω του $(7,6)$ εχουμε βρει καλυτερο μονοπατι εχοντας κοστος $f = 12.5 + 1 = 13.5 < 14.5$ , το $(7,7)$ δεν το ξαναεισαγουμε γιατι εχουμε κοστος $f = 13 > 12.5$ εαν το επισκευτουμαι μεσω του $(7,6)$ 
    \newline
     \vspace{-150cm}
     \vspace*{150cm}Fringe: [ $((5,4) ,13.5, $Κ$ )\hspace{0.3cm}\vert\hspace{0.3cm} ((5,8) ,13.5, $Δ$ )\hspace{0.3cm}\vert\hspace{0.3cm}((6,3) ,13.5, $Κ$ )\hspace{0.3cm}\vert\hspace{0.3cm} ((9,7) , 13.5 , $Κ$) \hspace{0.3cm}\vert\hspace{0.3cm}\\ ((8,6) , 13.5 , $K$)\hspace{0.3cm}\vert\hspace{0.3cm}((6,4) ,14, $Α$ ) \hspace{0.3cm}\vert\hspace{0.3cm}((7,8) ,14, $Δ$ ) \hspace{0.3cm}\vert\hspace{0.3cm}((0,9) ,14 , $Δ$)\hspace{0.3cm}\vert\hspace{0.3cm}((5,2) ,14.5, $Α$ ) \hspace{0.3cm}\vert\hspace{0.3cm}\\ ((3,8) ,14.5, $Δ$ )$ ]
     \newline
     \vspace{-150cm}
     \vspace*{150cm}
     Explored: [$S , (1,0) , (0,1) , (2,0) , (0,2) , (3,0) , (0,3) ,(4,0) , (3,1) , (1,3) ,(1,4) , (1,5) , (2,1) ,\\ (4,1) , (3,2) , (2,3) ,(2,5) ,(1,6) ,(3,3) ,(3,5) ,(2,6) , (1,7), (4,3), (4,5) ,(2,7) ,(4,6) ,(4,7) ,\\(2,2) ,(0,6) ,(0,7), (4,4) ,(5,5) ,(5,7) ,(3,6) ,(6,5) ,(6,7) ,(7,7) , (5,3) ,(3,7) ,(4,8) ,(0,8) ,\\ (6,6) ,(8,7) ,(7,6)$]
     \newline
     \newline
     $\bullet$\hspace{0.1cm}Αφαιρειται ο $(5,4)$ (που ειναι η Κατω ενεργεια του ρομποτ απο τον κομβο $(4,4$) ) , δεν εισαγουμε καποιον γειτονα γιατι για να παμε απο το $(5,4)$ στο $(5,3)$ και στο $(5,5)$ θα εχουμε μεγαλυτερο κοστος(για το $(5,3)$ $f = 12 + 4 = 16 >13$ και για το $(5,5)$ $f = 11 + 3 = 14 > 12$) , επισης δεν κανουμε update το $(6,4)$ που βρισκεται μεσα στο fringe γιατι εαν το κανουμε visit μεσω του $(5,4)$ θα εχουμε $f = 11 + 3 =14$
    \newline
     \vspace{-150cm}
     \vspace*{150cm}Fringe: [ $((5,8) ,13.5, $Δ$ )\hspace{0.3cm}\vert\hspace{0.3cm}((6,3) ,13.5, $Κ$ )\hspace{0.3cm}\vert\hspace{0.3cm} ((9,7) , 13.5 , $Κ$) \hspace{0.3cm}\vert\hspace{0.3cm}((8,6) , 13.5 , $K$)\hspace{0.3cm}\vert\hspace{0.3cm}\\((6,4) ,14, $Α$ ) \hspace{0.3cm}\vert\hspace{0.3cm}((7,8) ,14, $Δ$ ) \hspace{0.3cm}\vert\hspace{0.3cm}((0,9) ,14 , $Δ$)\hspace{0.3cm}\vert\hspace{0.3cm}((5,2) ,14.5, $Α$ ) \hspace{0.3cm}\vert\hspace{0.3cm}((3,8) ,14.5, $Δ$ )$ ]
     \newline
     \vspace{-150cm}
     \vspace*{150cm}
     Explored: [$S , (1,0) , (0,1) , (2,0) , (0,2) , (3,0) , (0,3) ,(4,0) , (3,1) , (1,3) ,(1,4) , (1,5) , (2,1) ,\\ (4,1) , (3,2) , (2,3) ,(2,5) ,(1,6) ,(3,3) ,(3,5) ,(2,6) , (1,7), (4,3), (4,5) ,(2,7) ,(4,6) ,(4,7) ,\\(2,2) ,(0,6) ,(0,7), (4,4) ,(5,5) ,(5,7) ,(3,6) ,(6,5) ,(6,7) ,(7,7) , (5,3) ,(3,7) ,(4,8) ,(0,8) ,\\ (6,6) ,(8,7) ,(7,6) ,(5,4)$]
     \newline
     \newline
    $\bullet$\hspace{0.1cm}Αφαιρειται ο $(5,8)$ (που ειναι η Δεξια ενεργεια του ρομποτ απο τον κομβο $(5,7$) )  και εισαγονται οι γειτονες του $((5,9) , 15 , $Δ$ )$ , δεν ξαναεισαγουμε το $(4,8) , (5,7)$ διοτι για να παμε απο το $(5,8)$ στο $(4,8)$ και στο $(5,7)$ θα εχουμε μεγαλυτερο κοστος(για το $(4,8)$ $f = 12 + 3 = 15 >13$ και για το $(5,7)$ $f = 12 + 2 = 14 > 12$) 
    \newline
     \vspace{-150cm}
     \vspace*{150cm}Fringe: [ $((6,3) ,13.5, $Κ$ )\hspace{0.3cm}\vert\hspace{0.3cm} ((9,7) , 13.5 , $Κ$) \hspace{0.3cm}\vert\hspace{0.3cm}((8,6) , 13.5 , $K$)\hspace{0.3cm}\vert\hspace{0.3cm}((6,4) ,14, $Α$ ) \hspace{0.3cm}\vert\hspace{0.3cm}\\((7,8) ,14, $Δ$ ) \hspace{0.3cm}\vert\hspace{0.3cm}((0,9) ,14 , $Δ$)\hspace{0.3cm}\vert\hspace{0.3cm}((5,2) ,14.5, $Α$ ) \hspace{0.3cm}\vert\hspace{0.3cm}((3,8) ,14.5, $Δ$ ) \hspace{0.3cm}\vert\hspace{0.3cm} ((5,9) , 15 , $Δ$ )$ ]
     \newline
     \vspace{-150cm}
     \vspace*{150cm}
     Explored: [$S , (1,0) , (0,1) , (2,0) , (0,2) , (3,0) , (0,3) ,(4,0) , (3,1) , (1,3) ,(1,4) , (1,5) , (2,1) ,\\ (4,1) , (3,2) , (2,3) ,(2,5) ,(1,6) ,(3,3) ,(3,5) ,(2,6) , (1,7), (4,3), (4,5) ,(2,7) ,(4,6) ,(4,7) ,\\(2,2) ,(0,6) ,(0,7), (4,4) ,(5,5) ,(5,7) ,(3,6) ,(6,5) ,(6,7) ,(7,7) , (5,3) ,(3,7) ,(4,8) ,(0,8) ,\\ (6,6) ,(8,7) ,(7,6) ,(5,4), (5,8)$]
     \newline
     \newline
    $\bullet$\hspace{0.1cm}Αφαιρειται ο $(6,3)$ (που ειναι η Κατω ενεργεια του ρομποτ απο τον κομβο $(5,3$) )  και εισαγονται οι γειτονες του $((6,2) , 15 , $Α$ ) , ((7,3) , 14 , $Κ$ )$ ,
    δεν κανουμε update το $(6,4)$ μεσα στο fringe , διοτι κανοντας το visti μεσω του $(6,3)$ εχουμε $f = 11 + 3 = 14$
    \newline
     \vspace{-150cm}
     \vspace*{150cm}Fringe: [ $((9,7) , 13.5 , $Κ$) \hspace{0.3cm}\vert\hspace{0.3cm}((8,6) , 13.5 , $K$)\hspace{0.3cm}\vert\hspace{0.3cm}((6,4) ,14, $Α$ ) \hspace{0.3cm}\vert\hspace{0.3cm}((7,8) ,14, $Δ$ ) \hspace{0.3cm}\vert\hspace{0.3cm}((0,9) ,14 , $Δ$)\hspace{0.3cm}\vert\hspace{0.3cm}\\ ((7,3) , 14 , $Κ$ )\hspace{0.3cm}\vert\hspace{0.3cm}((5,2) ,14.5, $Α$ ) \hspace{0.3cm}\vert\hspace{0.3cm}((3,8) ,14.5, $Δ$ ) \hspace{0.3cm}\vert\hspace{0.3cm} ((5,9) , 15 , $Δ$ )\hspace{0.3cm}\vert\hspace{0.3cm} ((6,2) , 15 , $Α$ ) $ ]
     \newline
     \vspace{-150cm}
     \vspace*{150cm}
     Explored: [$S , (1,0) , (0,1) , (2,0) , (0,2) , (3,0) , (0,3) ,(4,0) , (3,1) , (1,3) ,(1,4) , (1,5) , (2,1) ,\\ (4,1) , (3,2) , (2,3) ,(2,5) ,(1,6) ,(3,3) ,(3,5) ,(2,6) , (1,7), (4,3), (4,5) ,(2,7) ,(4,6) ,(4,7) ,\\(2,2) ,(0,6) ,(0,7), (4,4) ,(5,5) ,(5,7) ,(3,6) ,(6,5) ,(6,7) ,(7,7) , (5,3) ,(3,7) ,(4,8) ,(0,8) ,\\ (6,6) ,(8,7) ,(7,6) ,(5,4), (5,8) , (6,3)$]
     \newline
     \newline

    $\bullet$\hspace{0.1cm}Αφαιρειται ο $(9,7)$ (που ειναι η Κατω ενεργεια του ρομποτ απο τον κομβο $(8,7$)) , ειναι το \textbf{GOAL STATE} , (Θεωρω οτι δεν γινεται expanded) και ο αλγοριθμος τερματιζει!
    \newline
     \vspace{-150cm}
     \vspace*{150cm}Fringe: [ $((8,6) , 13.5 , $K$)\hspace{0.3cm}\vert\hspace{0.3cm}((6,4) ,14, $Α$ ) \hspace{0.3cm}\vert\hspace{0.3cm}((7,8) ,14, $Δ$ ) \hspace{0.3cm}\vert\hspace{0.3cm}((0,9) ,14 , $Δ$)\hspace{0.3cm}\vert\hspace{0.3cm} ((7,3) , 14 , $Κ$ )\hspace{0.3cm}\vert\hspace{0.3cm}\\((5,2) ,14.5, $Α$ ) \hspace{0.3cm}\vert\hspace{0.3cm}((3,8) ,14.5, $Δ$ ) \hspace{0.3cm}\vert\hspace{0.3cm} ((5,9) , 15 , $Δ$ )\hspace{0.3cm}\vert\hspace{0.3cm} ((6,2) , 15 , $Α$ ) $ ]
     \newline
     \vspace{-150cm}
     \vspace*{150cm}
     Explored: [$S , (1,0) , (0,1) , (2,0) , (0,2) , (3,0) , (0,3) ,(4,0) , (3,1) , (1,3) ,(1,4) , (1,5) , (2,1) ,\\ (4,1) , (3,2) , (2,3) ,(2,5) ,(1,6) ,(3,3) ,(3,5) ,(2,6) , (1,7), (4,3), (4,5) ,(2,7) ,(4,6) ,(4,7) ,\\(2,2) ,(0,6) ,(0,7), (4,4) ,(5,5) ,(5,7) ,(3,6) ,(6,5) ,(6,7) ,(7,7) , (5,3) ,(3,7) ,(4,8) ,(0,8) ,\\ (6,6) ,(8,7) ,(7,6) ,(5,4), (5,8) , (6,3)$]
     \newline
     \newline
     \newpage
    \textbf{\Large{\underline{Οποτε εχουμε:}} }\\
    \newline
    $\bullet$\textbf{\underline{47 κομβοι} επεκταθηκαν συνολικα(οσοι ειναι μεσα στο Explored Set)}\\
    \newline
    $\bullet$\textbf{\underline{Συνολικο Κοστος βελτιστης διαδρομης: 13.5}(Θεωρωντας οτι για να παμε στο G θελουμε κοστος 1) }\\
    \newline
    $\bullet$\textbf{\underline{Βέλτιστη Διαδρομή:} [Δ,Δ,Δ,Κ,Δ,Δ,Κ,Κ,Κ,Δ,Δ,Κ,Κ,Κ,Κ,Κ] }\\
    \newline
    \textbf{\underline{Παραδεκτες Συναρτησεις:}}\\
    \newline
    $\bullet$\textbf{\underline{Ευκλειδια Αποσταση:}}\\
    \newline
    \Large $h(x,y) = \sqrt{(x_2 - x_1)^{2} + (y_2 - y_1)^{2}}$\\
    \newline
    Η Ευκλειδια αποσταση ειναι παραδεκτη ευρετικη συναρτηση καθως οτι τιμη και να εχουν τα ζευγαρια συντεταγμενων $(x_1 , y_1) , (x_2 , y_2)$ το αποτελεσμα $h(x,y) \ge 0$ αφου εχουμε τα δυο τετραγωνα και το υποριζο ειναι σιγουρα μεγαλυτερο ή ισο του $0$ . Το αποτελεσμα της ευρετικης συναρτησης θα ειναι σιγουρα μικροτερο ή ισο με το πραγματικο κοστος καθως η ευκλειδια αποσταση ειναι το μηκος της ευθειας που εννωνει τα  δυο ζευγαρια σημειων και σιγουρα θα υπαρχουν εμποδια των οποιων το κοστος για να τα αποφυγουμε δεν το υπολογιζει η ευρετικη συναρτηση , αρα το κοστος που θα εκτιμησει θα ειναι μικροτερο ή ισο απο το πραγματικο.\\
    Οταν θα εχουμε φτασει στο Goal το $h(x,y) = 0$ αφου το αθροισμα των $2$ τετραγωνων θα μας δοσει $0$ . \\
    \newline
    Αρα με βαση τα παραπανω ισχυει η συνθηκη $0 \le h(x,y)\le $true cost\\
    \newline
    $\bullet$\textbf{\underline{$h(x,y) = max( |x_2 - x_1| , |y_2 - y_1| )$}}\\
    \newline
    Η παραπανω ευρετικη ειναι παραδεκτη καθως το αποτελεσμα της ειναι σιγουρα μεγαλυτερο ή ισο του μηδενος , αφου ειναι η μεγιστη τιμη δυο απολυτων τιμων. Επισης το αποτελεσμα της $0$ οταν ειμαστε στην κατασταση στοχου καθως θα ειναι η μεγιστη τιμη του $0 ,0$ . Τελος το αποτελεσμα της ειναι σιγουρα μικροτερο του πραγματικου κοστους καθως ειναι η μεγιστη τιμη του μηκους ή του πλατους για να παμε απο το σημειο $(x,y)$ στο goal , οι οποιες τιμες δεν λαμβανουν υποψη τους το κοστος των εμποδιων που υπαρχουν κατα μηκος της αποστασης που διανυουν.\\
    \newline
    \newline
    \textbf{\Large\textit{\underline{Πρόβλημα 4}}:}\\
    \newline
    Σε καθε περιπτωση θεωρουμαι τον αλγοριθμο αμφιδρομης αναζητησης και τα evaluation functions που εχουν παρασουαστει στο φροντηστηριο.\\
    \newline
    \textbf{a) \underline{BFS-DLS}}\\
    \newline
    Ο BFS με βαση τις διαφανειες εχει συναρτηση αποτιμησης για καθε κομβο την θετικη τιμη του βαθους στο οποιο βρισκεται , ενω ο DLS(που πρακτικα ειναι DFS απλος με ενα οριο βαθους)εχει την αρνιτικη τιμη του βαθους. Ετσι οπως ειναι δοσμενος ο αλγοριθμος η αναζητηση θα ξεκινησει πρωτα απο το Goal προς το Start κανοντας DLS. Αυτο σημαινει οτι σε καθε βημα οι κομβοι που θα επεκτινει ο DLS θα εχουν αρνητικη τιμη f(n) και ο αλγοριθμος θα επιλεγει να επεκτεινει αυτους αντι του start state του BFS που θα εχει τιμη 0(γιατι βρισκεται στο βαθος 0).Εαν το οριο βαθους που εχει θεσει ο DLS αρκει για να φτασει στο Goal τοτε θα βρεθει λυση κανοντας μονο DLS και ο αλγοριθμος θα ειναι πληρης.Εαν το οριο βαθους δεν αρκει τοτε σε αυτο το βαθος θα σταματισει η εκτελεση του DLS , ολοι οι κομβοι που θα εχει επεκτεινει θα εχουν βγει απο το αντοιστιχο frontier και ο αλγοριθμος θα συνεχισει κανωντας μονο BFS οπου και θα βρει την λυση(εαν ο παραγωντας διακλωδοσης ειναι πεπερασμενος) οταν συνιδιτοποιησει οτι ο κομβος τον οποιο επεκτινει υπαρχει στο reached set του DLS. Αρα και στις 2 περιπτωσεις ο αλγοριθμος ειναι πληρης. Φυσικα εαν ο χωρος καταστασεων ειναι απειρος τοτε και στις 2 περιπτωσεις ο αλγοριθμος δεν ειναι πληρης. \\ 
    \newline
    Ο αλγοριθμος δεν ειναι σιγουρο οτι θα βρει την βελτιση λυση και στις 2 περιπτωσεις καθως στην πρωτη που θα τρεξει μονο ο DLS (γιατι το οριο βαθους θα ειναι μεγαλυτερο ή ισο απο το βαθος της λυσης ), δεν ειναι βελτιστος αλγοριθμος. Το ιδιο ισχυει και για την δευτερη περιπτωση οπου ο DLS θα επεκτινει ενα συγκεκριμενο αριθμο κομβων , χωρις ομως να βρει την λυση , περιμενωντας μετα απο τον BFS για να βρει ενα path προς καποιον κομβο που εχει επεκτινει ο DLS, το οποιο ομως δεν ειναι σιγουρο οτι θα ειναι το βελτιστο καθως μπορει να υπαρξει καποιο αλλο με λιγοτερο κοστος.\\ 
    \newline
    Για να ελεγξουμε αποδοτικα  εαν οι δυο αναζητησεις συναντιουνται , θα εχουμε δυο set ενα για την εκτελεση απο την αρχη προς το τελος και ενα απο το τελος προς την αρχη και καθε κομβο που επεκτινουμε θα βαζουμε τα παιδια του στο set του αλγοριθμο που τρεχει, αρα αρκει να ελεγχουμε εαν ο κομβος που επεκτεινει στην τρεχουσα επαναληψη ο DLS ή ο BFS βρισκεται στο reached set του αλλου ,αυτος ο ελεγχος θα γινεται σε Ο(1) χρονο γιατι η δομη που χρησμοποιουμε ειναι set.\\ 
    \newline
    \textbf{b) \underline{IDS-DLS}}\\
    \newline
    Στην περιπτωση του IDS-DLS παλι θα ξεκινησουμε κανοντας DLS(γιατι δεν ικανοποιειται η συνθηκη if που ελεγχει εαν το πρωτο στοιχειο απο το frontier1 ειναι μικροτερο απο αυτο του frontier2) μεχρι να βρουμε το start node απο το οποιο θα ξεκινουσε ο IDS. Αρα εαν το οριο βαθους που εχει θεσει ο DLS ειναι αρκετο ετσι ωστε να φτασουμε στο goal τοτε ο αλγοριθμος θα ειναι πληρης , βρισκοντας μονοπατι . Εαν το οριο βαθους δεν αρκει τοτε το frontier2 θα ειναι αδειο και θα εξαγωνται κομβοι μονο απο το frontier1 μεχρι ο αλγοριθμος IDS να συναντισει καποιον expanded απο τον DLS. Και σε αυτην την περιπτωση ο αλγοιρθμος θα βρει λυση αρα ειναι πληρης. Υποθετουμε φυσικα οτι ο χωρος καταστασεων δεν ειναι απειρος , αλλιως δεν θα βρισκαμαι λυση σε καμια απο τις δυο περιπτωσεις. Επισης υποθετουμε οτι ο παραγωντας διακλαδωσης ειναι πεπερασμενος αλλιως ο IDS δεν θα βρει λυση στην δευτερη περιπτωση.\\
    \newline
    Ο αλγοριθμος δεν ειναι σιγουρο οτι θα βρει την βελτιση λυση και στις 2 περιπτωσεις καθως στην πρωτη που θα τρεξει μονο ο DLS (γιατι το οριο βαθους που θα εχουμε επιλεξει θα ειναι μεγαλυτερο ή ισο απο το βαθος της λυσης) , δεν ειναι βελτιστος αλγοριθμος. Το ιδιο ισχυει και για την δευτερη περιπτωση οπου ο DLS θα επεκτινει ενα συγκεκριμενο αριθμο κομβων , χωρις ομως να βρει την λυση , περιμενωντας μετα απο τον IDS για να βρει ενα path προς καποιον κομβο που εχει επεκτινει ο DLS, το οποιο ομως δεν ειναι σιγουρο οτι θα ειναι το βελτιστο καθως και ο IDS κανει DFS αλλα καθε φορα πηγενει ενα βαθος παραπερα ξεκινωντας καθε φορα απο την αρχη , αρα δεν ειναι σιγουρο οτι στο σημειο που θα συναντιθουν οι δυο αλγοριθμοι οτι αυτο θα αντιστιχει στο βελτιστο μονοπατι.\\
    \newline
    Για να ελεγξουμε αποδοτικα  εαν οι δυο αναζητησεις συναντιουνται , θα εχουμε δυο set ενα για την εκτελεση απο την αρχη προς το τελος και ενα απο το τελος προς την αρχη και καθε κομβο που επεκτινουμε θα  βαζουμε τα παιδια του στο set του αλγοριθμο που τρεχει, αρα αρκει να ελεγχουμε εαν ο κομβος που επεκτεινει στην τρεχουσα επαναληψη ο DLS ή ο IDS βρισκεται στο reached set του αλλου ,αυτος ο ελεγχος θα γινεται σε Ο(1) χρονο γιατι η δομη που χρησιμοποιουμε ειναι set.\\ 
    \newline
    \textbf{c) \underline{Α$^{*}$-DLS}}\\
    \newline
    Στην περιπτωση του Α$^{*}$-DLS παλι θα ξεκινησουμε κανοντας DLS και θα συνεχισουμε μεχρι να βρουμε την κατασταση στοχου  ή μεχρι να φτασουμε στο οριο βαθους χωρις να βρουμε το goal node με αποτελεσμα στην συνεχεια να τρεξει ο Α$^{*}$ . Στην πρωτη περιπτωση ο αλγοριθμος θα ειναι πληρης καθως εαν ο χωρος καταστασεων δεν ειναι απειρος ο DLS θα φτασει στην κατασταση στοχου του , χωρις ομως να μας εγκυαται οτι η λυση που θα βρει θα ειναι η βελτιστη. Στην δευτερη περιπτωση ,δηλαδη σε αυτην που ο DLS θα φτασει μεχρι το οριο βαθους και θα σταματησει ,τοτε θα  ξεκινησει να εκτελειται ο Α$^{*}$ μεχρι να βρεθει καποιο node που να ειναι στο reached set του DLS. Στην περιπτωση αυτη ο αλγοριθμος θα ειναι πληρης αλλα οχι βελτιστος , διοτι ο DLS θα εχει ηδη επεκτινει εναν αριθμο κομβων και ο  Α$^{*}$ με το που πεσει σε εναν απο αυτους θα σταματησει αγνωωντας την υπαρξει αλλων πιο βελτιστων μονοπατιων.Παλι υποθετουμε οτι για να ειναι πληρης δεν εχουμε απειρο χωρο καταστασεων.\\
    \newline
    Οπως και στις προηγουμενες περιπτωσεις αλγοριθμων για να ελεγξουμε αποδοτικα  εαν οι δυο αναζητησεις συναντιουνται , θα εχουμε δυο set ενα για την εκτελεση απο την αρχη προς το τελος και ενα απο το τελος προς την αρχη και καθε κομβο που επεκτινουμε θα βαζουμε τα παιδια του στο set του αλγοριθμο που τρεχει , αρα αρκει να ελεγχουμε εαν ο κομβος που επεκτεινει στην τρεχουσα επαναληψη ο DLS ή ο Α$^{*}$ βρισκεται στο reached set του αλλου ,αυτος ο ελεγχος θα γινεται σε Ο(1) χρονο γιατι η δομη μας ειναι set.\\ 
    \newline
    \textbf{d) \underline{Α$^{*}$-Α$^{*}$}}\\
    \newline
    Στην περιπτωση που κανουμε Α$^{*}$ και απο τις δυο μεριες , και ο χωρος καταστασεων δεν ειναι απειρος , ο αλγοριθμος αμφιδρομης αναζητησς θα ειναι και πληρης αλλα και βελτιστος. Θα ειναι πληρης με την προυποθεση οτι εχουμε επιλεξει μια παραδεκτη ευρετικη συναρτηση. Ετσι οι δυο αλγοριθμοι θα συναντηθουν σιγουρα σε καποιον κομβο και το μονοπατι που θα σχηματιστει θα αποτελει και την βελτιστη λυση καθως τρεχουμαι Α$^{*}$ που ειναι ενας βελτιστος αλγοριθμος και απο τις 2 μεριες , αρα ειναι σιγουρο οτι οι κομβοι που θα επεκτινονται σε καθε βημα και απο τις δυο μεριες θα μας φερνουν ολο και πιο κοντα στην βελτιστη λυση.\\
    \newline
    Οπως και στις προηγουμενες περιπτωσεις αλγοριθμων για να ελεγξουμε αποδοτικα  εαν οι δυο αναζητησεις συναντιουνται , θα εχουμε δυο set ενα για την εκτελεση απο την αρχη προς το τελος και ενα απο το τελος προς την αρχη και καθε κομβο που επεκτινουμε θα βαζουμε τα παιδια του στο set του αλγοριθμο που τρεχει , αρα αρκει να ελεγχουμε εαν ο κομβος που επεκτεινει στην τρεχουσα επαναληψη ο Α$^{*}$ απο την αρχη προς το τελος ή ο Α$^{*}$ απο το τελος προς την αρχη βρισκεται στο reached set του αλλου ,αυτος ο ελεγχος θα γινεται σε Ο(1) χρονο γιατι η δομη μας ειναι set.\\ 
    \newline
    
\end{document} 
